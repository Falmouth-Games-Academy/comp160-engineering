\documentclass{scrartcl}

\usepackage[hidelinks]{hyperref}
\usepackage[none]{hyphenat}

\title{Essay Proposal}
\subtitle{COMP110 - Computer Architecture Essay}

\author{Hannah Utley}

\begin{document}

\maketitle

\section*{Evidence-Based Practice for Designing Accessible Games:\\
 How can mainstream games accommodate for the Visually Impaired?}

{Within software engineering, the need for more evidence-based practice has been argued, to tackle the trend for bloated and unmaintainable systems \cite{softwaredevelopers}. In the gaming industry, software developers are under pressure to meet tight deadlines, often sacrificing the quality of the design and implementation of code, as such, accessibility is often given little priority \cite{D.McPheron}. Whilst other software, such as office computer applications and web applications, have been adapted successfully for the visually impaired, the gaming industry is comparatively poor.  Developing games to meet the many varying cognitive, physical, and visual needs of the population is a complicated and often expensive task. Mainstream games reliance on impressive visuals makes accessibility for the visually impaired, far more complex than other software applications. However, several creative technologies and strategies have been designed to improve games for the visually impaired, using music \cite{Lotto}, specially designed gaming devices \cite{Ohtsuka}, and using conversion of images into sounds \cite{Marshall}. This paper aims to explore existing games designed specifically for the visually impaired and how mainstream games can learn from these examples, obstacles that software developers face in constructing accessible games and how academia can help prepare students to assist in improving accessibility.}

\section*{Paper 1}
\begin{description}
\item[Title:] Video Gaming Accessibility
\item[Citation:] \cite{D.McPheron}
\item[Abstract:] ``Video games are becoming less and less usable by people with disabilities every day. These disabilities include cognitive, visual, and mobility. The purpose of this paper is to discuss the necessity of making video games accessible. This will be done through looking at benefits, current work for all disabilities, and how it's looked at by the law. The topic of video gaming accessibility is especially important in today's age with the disabled population rising. It is important to not let the disabled be left out of the enjoyment of video games.''
\item[Web link:] \url{http://ieeexplore.ieee.org.ezproxy.falmouth.ac.uk/document/7272966/6}
\item[Comments:] This paper gives a generalised view of the importance, as well as the current lack of, accessibility in gaming. 
\end{description}

\section*{Paper 2}
\begin{description}
\item[Title:] Computer Games and Visually Impaired People
\item[Citation:] \cite{Archambault}
\item[Abstract:] ``Accessibility of computer games is a Challenge. Indeed, making accessible a computer game is much more difficult than making accessible a desktop application. In this paper we first define game accessibility. Then we present a number of papers published in the last decade: spe- cific games (audiogames, tactile games etc), games designed for all, and a few works about game accessibility. Then we will describe the work that we are currently carry out in order to propose a framework allowing mainstream games accessibility."
\item[Web link:] http://citeseerx.ist.psu.edu/viewdoc/summary?doi=10.1.1.76.562
\item[Comments:] This paper explains the challenges and importance of desgining accessible games for the visually impaired.
\end{description}

\section*{Paper 3}
\begin{description}
\item[Title:] Guidelines of Serious Game Accessibility for the Disabled
\item[Citation:] \cite{H.J.Park}
\item[Abstract:] ``Web accessibility/mobile app accessibility for the disabled has been studied for the past 10 years. The legislation of web accessibility guidelines can guarantee easy access to web contents of the disabled but not game playing, because web/ mobile app contents consists of rather simple information compared with game contents. Game contents include many PCs (player characters)/NPCs (non PCs) and the conflicts among them, as well as upgrade of a player character by completion of a quest in competition. Therefore, it is necessary to analyze and classify game accessibility so as to make the game accessibility guidelines."
\item[Web link:] http://ieeexplore.ieee.org.ezproxy.falmouth.ac.uk/document/6579380/
\item[Comments:] This paper has a useful guideline for gaming accessibility. 
\end{description}

\section*{Paper 4}
\begin{description}
\item[Title:] Video and computer games for people with vision impairment
\item[Citation:] \cite{rnib}
\item[Abstract:] ``“Until I went to college, I didn’t even know how to use a computer properly,” says gamer Ian McNamara. “Then when my friends introduced me computer games for blind people, I thought they were having a laugh at first. That’s how underground it was. If I hadn’t gone to the Royal National College for the Blind, I’d never have found the blind gaming community.""
\item[Web link:] http://www.rnib.org.uk/nb-online/video-computer-games-people-vision-impairment
\item[Comments:] This is an article on the RNIB website, useful as evidence that visually impaired people can and do enjoy video games, and therefore stresses the importance of making it possible to play more mainstream games.
\end{description}

\section*{Paper 5}
\begin{description}
\item[Title:] ShadowRine: Accessible game for blind users, and accessible action RPG for visually impaired gamers
\item[Citation:] \cite{Matsuo}
\item[Abstract:] ``Though some games for visually impaired persons have been developed, most of games that use only auditory information present challenges for sighted persons. Moreover, unfortunately, it is still difficult for visually impaired persons to play the same game with sighted persons and for sighted and visually impaired persons to share a common subject. Thus, we developed a barrier-free game that both sighted and visually impaired persons can play using their dominant senses including visual, auditory and tactile senses"
\item[Web link:] http://ieeexplore.ieee.org.ezproxy.falmouth.ac.uk/document/7844667/
\item[Comments:] This paper explores a case study of ShadowRine and the successes and failrues in trying to accomodate the visually impaired.
\end{description}

\section*{Paper 6}
\begin{description}
\item[Title:] Making the mainstream accessible: redefining the game
\item[Citation:] \cite{Atkinson}
\item[Abstract:] ``This work in progress is focused on adapted design of a set of games for people with cerebral palsy. In the specification all the stakeholders (family, therapeutics and monitors) have participated. In the games set 4 levels of complexity have been proposed, and also the different types of interaction: touchscreen, push buttons, etc. The evaluation of the 4 prototypes will be held including not only testing the proper functionality of the games, but ensuring the accessibility and usability standards."
\item[Web link:] http://dl.acm.org.ezproxy.falmouth.ac.uk/
\item[Comments:] This paper focuses on mainstream games and the need to make them accessible, whilst I have found several creative games for the visually impaired, I have found little evidence of accessibilty in mainstream games.
\end{description}

\section*{Paper 7}
\begin{description}
\item[Title:] AccTrace: Accessibility in Phases of Requirements Engineering, Design, and Coding Software
\item[Citation:] \cite{Branco}
\item[Abstract:] ``Providing accessible web products is a challenge. In particular, the implementation of requirements is still a problem for many developers who may not have the necessary skills to perform this task, including difficulty in interpreting reference documents. This paper presents the AccTrace, a CASE tool (Eclipse plugin) that uses an ontology to specify the technical implementation of accessibility and promotes traceability of accessibility requirements from conception to the coding phases. This may give the developer useful information for the construction of accessible product."
\item[Web link:] http://ieeexplore.ieee.org.ezproxy.falmouth.ac.uk/document/6976693/
\item[Comments:]  This paper takes a deeper loom at software engineering for accessibility.
\end{description}

\section*{Paper 8}
\begin{description}
\item[Title:] Kinaptic - Techniques and insights for creating competitive accessible 3D games for sighted and visually impaired users
\item[Citation:] \cite{Grabski}
\item[Abstract:] ``We present the first accessible game that allows a fair competition between sighted and blind people in a shared virtual 3D environment. We use an asymmetric setup that allows touchless interaction via Kinect, for the sighted player, and haptic, wind, and surround audio feedback, for the blind player. We evaluated our game in an in-the-wild study. The results show that our setup is able to provide a mutually fun game experience while maintaining a fair winning chance for both players. Based on our study, we also suggest guidelines for future developments of games for visually impaired people that could help to further include blind people into society."
\item[Web link:] http://ieeexplore.ieee.org.ezproxy.falmouth.ac.uk/document/7463198/
\item[Comments:] This paper highlights a case study of a 3D game, specfically how to ensure a fair competitive game between sighted and visually impared players. 
\end{description}

\section*{Paper 9}
\begin{description}
\item[Title:] Designing Software that is Accessible to Individuals with Disabilities
\item[Citation:] \cite{doittwo}
\item[Abstract:] ``Accessibility guidelines for software."
\item[Web link:] http://www.washington.edu/doit/designing-software-accessible-individuals-disabilities
\item[Comments:] This paper highlights the importance of accessible gaming.
\end{description}

\section*{Paper 10}
\begin{description}
\item[Title:] Accessible Game Culture Using Inclusive Game Design - Participating in a Visual Culture That You Cannot See
\item[Citation:] \cite{Wihelmsson} 
\item[Abstract:] ``In this paper, we present the result of an experiment, in which we compare the gaming experience between sighted players and visually impaired players playing the same game. Specifically we discuss whether they experience the same story construed from the plot elements that are either manifested by audio and graphics in the case of sighted players or primarily by audio in the case of visually impaired players. To this end, we have developed a graphical point-and-click adventure game for iOS and Android devices. The game has been designed to provide players with audio feedback that enables visually impaired players to interact with and experience the game, but in a manner that does not interfere with the overall appearance and functionality of the game, i.e. a design that is fully inclusive to both groups of players and that is as invisible for sighted players as possible without hindering visually impaired players to share the same gaming experience when it comes to story content. The study shows that the perception of the story was almost identical between the two groups. Generally it took visually impaired players a little longer to play the game but they also seem to listen more carefully to the dialogue and hence also build a slightly deeper understanding of the characters."
\item[Web link:] http://ieeexplore.ieee.org.ezproxy.falmouth.ac.uk/document/7295764/
\item[Comments:] This paper explores the experiences of visually impaired playing a game alongside sighted players.
\end{description}

\section*{Paper 11}
\begin{description}
\item[Title:] Serious 3D gaming research for the vision impaired
\item[Citation:] \cite{Marshall}
\item[Abstract:] ``This paper describes “A walk in the park” which is a serious game designed and developed for performing auditory-vision sensory substitution (AVSS) research for the visually impaired. Until now, evaluating AVSS solutions required the creation of a sensory substitution device (SSD) to test a strategy for converting images into sounds. This process often takes years to complete. 3D game engines provide a research and training platform that is capable of quickly implementing and testing different image-to-sound conversion strategies in a virtual world environment."
\item[Web link:] http://ieeexplore.ieee.org.ezproxy.falmouth.ac.uk/document/7454547/
\item[Comments:] This paper shows a case study of a game which converts images into sounds in order to be accessible to the visually impaired. This is an interesting technique and I would like to learn more about it, especially in terms of our group project.
\end{description}

\section*{Paper 12}
\begin{description}
\item[Title:] Issues in student valuing of software engineering best practices
\item[Citation:] \cite{Frezza}
\item[Abstract:] ``This paper outlines the need for valuing of software engineering skills as a means to improve software engineering education. It presents a brief introduction to affective domain learning, and a survey of the education literature on software engineering skills related to software testing and quality assurance, which suggests that the competencies and skills needed extend beyond cognitive-domain learning. It then proposes a means for studying student valuing of these `best practice' skill areas"
\item[Web link:] http://ieeexplore.ieee.org.ezproxy.falmouth.ac.uk/document/7757556/
\item[Comments:] This paper highlights some of the issues the industry faces when graduates have not taken software enginerring practices seriously during the course of their study, this is an obstacle to improving accessibility.
\end{description}

\section*{Paper 13}
\begin{description}
\item[Title:] Virtual Stage: An Immersive Musical Game for People with Visual Impairment
\item[Citation:] \cite{Lotto}
\item[Abstract:] ``Musical games help in motor and cognitive development and provide pleasure because of their playfulness and challenge. However, visually impaired people are unable to interact with conventional music games, because audiovisual aspects are inaccessible to this public. The objective of this study was to investigate if three-dimensional sounds could guide blind users during interaction with a tablet music game. As a proof of concept, a musical game for Android was developed with binaural audio techniques. Three fully blind users evaluated the game. The results show that the developed game was able to immerse the users in a virtual environment."
\item[Web link:]http://ieeexplore.ieee.org.ezproxy.falmouth.ac.uk/document/7785850/
\item[Comments: ]Using music to convert/replace images for the visually impaired is an interesting solution and is something that may feasibly be incorporated into our group game project.
\end{description}

\section*{Paper 14}
\begin{description}
\item[Title:] A New Game Device Using Body-Braille for Visually Impaired People.
\item[Citation:] \cite{Ohtsuka}
\item[Abstract:] ``During the last five years, we have been developing the Body-Braille system which uses 6 vibration motors corresponding to the units of Braille. Disabled people can wear this system and sense Braille characters by vibration patterns on any part of the body. We have performed several experiments in which disabled people could get needed support for daily life through Body-Braille. Since we are developing a new type of equipment, we chose to make one of its main applications a game. Our new equipment is so small that it is very easy to carry it anywhere, such as on a train. Disabled people can play a game using the equipment while traveling around. The software of the game application employs a table based structure so that anyone can create a new game application with very little programming work. As a result, we have successfully developed the software for game creation."
\item[Web link:] http://ieeexplore.ieee.org.ezproxy.falmouth.ac.uk/document/4148859/
\item[Comments:] This paper details the design and experimentation for a game device for the visually impaired. The device uses body-braille and is an interesting solution to gaming controllers. 
\end{description}

\section*{Paper 15}
\begin{description}
\item[Title:] What Should We Teach New Software Developers? Why?
\item[Citation:] \cite{softwaredevelopers}
\item[Abstract:] ``Computer science must be at the center of software systems development. If it is not, we must rely on individual experience and rules of thumb, ending up with less capable, less reliable systems, developed and maintained at unnecessarily high cost. We need changes in education to allow for improvements of industrial practice."
\item[Web link:] http://cacm.acm.org/magazines/2010/1/55760-what-should-we-teach-new-software-developers-why/fulltext
\item[Comments:] Stroustrup writes about the gap between academia and the industry in software development, the article shows some of the obstacles the industry faces. The current issue of bloated and unmaintainable systems makes it harder to prioritise accessibility, however, it may be that prioritising accessibility from the start may benefit developers in developing clearer and more maintainable systems.
\end{description}

\bibliographystyle{ieeetran}
\bibliography{initial_references}


\end{document}