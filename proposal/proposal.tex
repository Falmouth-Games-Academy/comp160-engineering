\documentclass{scrartcl}

\usepackage[hidelinks]{hyperref}
\usepackage[none]{hyphenat}

\title{Essay Proposal}
\subtitle{COMP160 - Software Engineering Essay}

\author{Nathan-David Coplestone}

\begin{document}

\maketitle

\section*{Engineering for Accessibility}

My essay will be on: How have AAA games adapted to the hearing impared and how can they improve? I will start this essay by looking at how different games and systems have been adapted to accomodate the deaf and hard of hearing. I will then look at how effective these methods are and how frequently they are used within games in the industry. I will then go on to look at different methods that could be used to improve on how the hearing impared can be accomodated for.

\section*{Paper 1}

\begin{description}
\item[Title:] Challenges and Lessons Learned in Teaching Software Engineering and Programming to Hearing-Impaired Students
\item[Citation:] \cite{Distante}
\item[Abstract:] ``Teaching academic courses to students with disabilities is a challenging task, particularly for academics who are presented with the teaching requirements and needs that this implies, for the first time. Courses in the field of engineering and computer science, by requiring a lot of handson practices and teamwork, further exacerbate the situation as how to provide an effective learning experience for these disabled students. This situation requires a higher-level commitment than normal, from both the teachers and students. This paper presents the experience gained from teaching courses that involved hearing-impaired students of an undergraduate software engineering and a programming language course in two different universities. Some of the challenges faced by both instructors and the students are identified and some possible solutions are described.''
\item[Web link:] \url{http://ieeexplore.ieee.org.ezproxy.falmouth.ac.uk/document/4271623/}
\item[Full text link:] \url{http://ieeexplore.ieee.org.ezproxy.falmouth.ac.uk/xpls/icp.jsp?arnumber=4271623}
\item[Comments:] I found this article while searching for papers on hearing impared people and how they relate to the industry, this is relevant to my essay as it describes effective ways to teach hearing impared people.
\end{description}

\section*{Paper 2}
\begin{description}
\item[Title:] The presentation of continuous speech with synchronous printed text
\item[Citation:] \cite{Sargent}
\item[Abstract:] ``A communications problem encountered by most hearing impared people is their inability to understand spoken English. Since present technology appears unable to eliminate this speech perception problem, it is hoped that a better understanding of the relationship between printed and spoken English will permit the hearing impared person to better use their residual hearing. To aid in such instruction, a system is being developed which permits a fully synchronized presentation of recorded speech with its corresponding printed text. A series of computer algorithms are employed to segment the speech signal into syllable-like units, and to separate the corresponding printed text into syllables. The resulting data are then combined on a syllable-by-syllable basis, and any synchronization errors are corrected through operator intervention prior to the creation of the synchronized speech/text material for the classroom."
\item[Web link:] \url{http://ieeexplore.ieee.org.ezproxy.falmouth.ac.uk/document/1170669/}
\item[Comments:] I found this article while searching for papers on hearing impared people and how they relate to the industry, this is relevant to my essay as it describes effective ways to teach hearing impared people.
\end{description}

\section*{Paper 3}
\begin{description}
\item[Title:] An Analysis of Information Conveyed through Audio in an FPS Game and Its Impact on Deaf Players Experience
\item[Citation:] \cite{Coutinho}
\item[Abstract:] ``Mainstream games usually lack support for accessibility to deaf and hard of hearing people. The popular FPS game Half-Life 2 is an exception, in that it provides well constructed closed captions to players. In this paper, we performed a semiotic inspection on Half-Life 2, seeking to identify which strategies were used to convey information through audio. We also evaluated how the loss of information in each of them may impact players' experience. Our findings reveal that six different strategies are used and how they may compromise player experience."
\item[Web link:] \url{http://ieeexplore.ieee.org.ezproxy.falmouth.ac.uk/document/6363218/}
\item[Comments:] I found this article while looking at how specifically, video games can be portrayed to hearing impared people. This is relevant to my essay as it shows how each players experice is differently impacted.
\end{description}

\section*{Paper 4}
\begin{description}
\item[Title:] Vocabulary acquisition for deaf readers using augmented technology
\item[Citation:] \cite{Jones}
\item[Abstract:] ``Learning how to read for deaf and hard-of-hearing children poses a few challenges that have been explored over the years by various researchers [13]. We propose an innovative solution that may enhance the way deaf and hard-of-hearing children learn to read through the use of a head-mounted-display (HMD). Through the HMD, deaf and hard-of-hearing children will be able look up unfamilar printed words from a varity of sources and have an American Sign Language (ASL) video definition play back for them."
\item[Web link:] \url{http://ieeexplore.ieee.org.ezproxy.falmouth.ac.uk/document/6799461/}
\item[Comments:] I found this paper when looking for ways to help the hard of hearing to understand different information. This is useful for my essay as it suggests ways that different technology can be integrated into different video games to help hearing impared people.
\end{description}

\section*{Paper 5}
\begin{description}
\item[Title:] Designing a game generator as an educational technology for the deaf learners
\item[Citation:] \cite{Bouzid}
\item[Abstract:] ``Children today grow up in an exciting and changing world where the web technology, Internet, mobile phones, video games and computers surround all aspects of their daily lives. These digital technologies, in particular video games, have provided new opportunities for these kids to play, communicate with others, foster their imagination and also enhance their educational performance. Unfortunately, this can be a little more different for deaf and hard of hearing children, whose native language is Sign Language (SL). The most existing video games are not geared to the unique need of these hearing disabled: there have been, even timidly, few attempts that focusing on the development of accessible educational games that could meet the requirements necessary for deaf gamers and improve their educational potential. In this context, we present in this paper a design for an interactive game generator of an accessible and effective educational game target towards the deaf and hard of hearing children. The game is called MemoSign, it federates the use of the famous memory game and a 3D signing avatar to support learning the SL notation system SignWriting. The main goal of designing such game generator is to help teachers, even parents, to create many instances of MemoSign, in few steps without any programming language knowledge, with a view to foster and promote the vocabulary acquisition for deaf learners in both signed and spoken languages."
\item[Web link:]  \url{http://ieeexplore.ieee.org.ezproxy.falmouth.ac.uk/document/7426914/}
\item[Full text link:] \url{http://ieeexplore.ieee.org.ezproxy.falmouth.ac.uk/xpls/icp.jsp?arnumber=7426914}
\item[Comments:] I found this paper when looking for ways to help the hard of hearing to understand different information. This is useful for my essay as it gives a good insight into what has already been tried and what hasn't.
\end{description}

\section*{Paper 6}
\begin{description}
\item[Title:] Sound Preferences of Persons with Hearing Loss Playing an Audio-based Computer Game
\item[Citation:] \cite{Hiraga}
\item[Abstract:] ``We performed an experiment to investigate differences between persons with and without hearing losses when playing a novel audio-based game on a tablet computer, and how persons with hearing losses appreciated the game when they played it with three different types of sound material---speech, music, or mixed speech and music. We analyzed game log files and participants' self-assessments and obtained results showing that there were significant differences between the two participant groups in terms of whether they completed the game. Moreover, the hearing loss group showed a preference for music among the three types of sounds and for the game itself. The two groups listened to music in different ways: hearing participants worked with the music material differently compared with other two types of material, implying that music is the most difficult among the three types. The hearing loss group showed preference for the music only-condition, which is consistent with the results from preliminary experiments we have done. We suggest that this novel game has the potential to improve the listening ability of persons with a hearing loss."
\item[Web link:]  \url{http://doi.acm.org.ezproxy.falmouth.ac.uk/10.1145/2505483.2505489}
\item[Comments:] I found this paper when looking for ways to help the hard of hearing to understand different information. This is useful for my essay as it gives a good insight into what has already been tried and what hasn't.
\end{description}

\bibliographystyle{ieeetran}
\bibliography{references}

\end{document}
