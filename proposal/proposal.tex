\documentclass{scrartcl}

\usepackage[hidelinks]{hyperref}
\usepackage[none]{hyphenat}

\title{Essay Proposal}
\subtitle{COMP160 - Software Engineering Essay}

\author{James Collins}

\begin{document}

\maketitle

\section*{Localisation in Software Engineering}

My essay will be on: Is the extra Cost, Time and Manpower usage Worth the benefits of Software Localisation.

% Add details as appropriate.

\section*{Paper 1}
% This is an example! Replace the details with a paper relevant to your chosen topic.
\begin{description}
\item[Title:]Software Internationalization and Localization: An Industrial Experience.
\item[Citation:] \cite{Xia2013software}
\item[Abstract:] Software internationalization and localization are important steps
 in distributing and deploying software to different regions of the world. 
Internationalization refers to the process of reengineering a system such
 that it could support various languages and regions without further modification.
 Localization refers to the process of adapting an internationalized software 
for a specific language or region. Due to various reasons, many large legacy 
systems did not consider internationalization and localization at the early stage of development. In this paper, we present our experience on, and propose 
a process along with tool supports for software internationalization and localization. We reengineer a large legacy commercial financial system called PAM 
of State Street Corporation, which is written in C/C++, containing 30 different modules, and more than 5 millions of lines of source code. We propose a 
source code ranker that recovers important source code to be analyzed. Based on this code, we extract general patterns of the source code that need
 to be reengineered for internationalization. We divide the patterns into 2 categories: convertible patterns and suspicious patterns. 
\item[Web link:] \url{http://ieeexplore.ieee.org/document/6601827/}
\item[Full text link:] \url{http://ieeexplore.ieee.org/stamp/stamp.jsp?arnumber=6601827}
\item[Comments:]This paper is interesting and relevant as it shows the difficulties of localising a piece of software that wasn't designed for localisation, and the amount of effort required to do so.
\end{description}

\section*{Paper 2}
\begin{description}
\item[Title:] Study on International Software Localization Testing
\item[Citation:] \cite{zhao2010study}
\item[Abstract:]Localization testing ensures that localized software meets the requirements of local language, market and user’s habits, it is an important part of international software testing. In, this paper, based on localization testing theory and applications, the concept, content and purpose of localization testing are described first, then three commonly used testing models have been compared, moreover, the testing environment setting, the testing implementation strategies, the localization testing phases and their flows has been analyzed and discussed.
\item[Web link:] http://ieeexplore.ieee.org/document/5718389/
\item[Full text link:] http://ieeexplore.ieee.org/stamp/stamp.jsp?arnumber=5718389
\item[Comments:] This paper puts forward a framework for testing how successful the localisation of a piece of software has been.
\end{description}

\section*{Paper 3}
\begin{description}
\item[Title:]Software localization for Internet software, issues and methods.
\item[Citation:] \cite{collins2002software}
\item[Abstract:] For use by a global audience, Web sites must be adapted to many local requirements. This article examines key issues in such adaptation (termed localization), considers the costs and specific aspects of software that must be localized, and presents an approach for analyzing and documenting software localization. The article is based on a review of relevant literature, meetings with localization industry representatives, and an ongoing participant observation in a global telehealth company. Examples from the company illustrate the localization issues and their possible outcomes or solutions.
\item[Web link:]http://ieeexplore.ieee.org/document/991367/
\item[Full text link:] http://ieeexplore.ieee.org/stamp/stamp.jsp?arnumber=991367
\item[Comments:] This article puts forward some interesting information on the costs of localising different aspects of software. Although it focuses on web design they could be carried across into other areas of software design. 
\end{description}

\section*{Paper 4}
\begin{description}
\item[Title:] Software localization: Translation memory using
\item[Citation:] \cite{sviridova2009software}
\item[Abstract:] This paper deals with problems of software localization and globalization. Translation memory advantages and disadvantages are discussed. As result translation memory was created and translation of documentation performed.
\item[Web link:]http://ieeexplore.ieee.org/document/4839918/
\item[Full text link:] http://ieeexplore.ieee.org/stamp/stamp.jsp?arnumber=4839918
\item[Comments:]This article puts forward a method that makes the localisation process easier and less time consuming. This argues that actually it could be worth the time and manpower required.
\end{description}

\section*{Paper 5}
\begin{description}
\item[Title:] Software Localization kit for Indian mass
\item[Citation:] \cite{tomar2014software}
\item[Abstract:]Software Localization includes adaptation, translation and customization of software product for the specific target market where it would be used and sold. Thus localization deals with a specific target locale or cultural convention. Localization related project development being discussed in the present paper is based on a localization architectural framework which provides a complete computing environment in local languages. Based on the proposed framework a service oriented localization tool is developed in java which facilitates the localization of java based applications from selected source language to the specific target language. The purpose of the paper is to describe the service oriented localization framework and its applicability in the form of localization tool. The paper further discusses the complete implementation details of the localization tool. Such development work may helps in performing simulation study for interoperability issues in localization process.
\item[Web link:] http://ieeexplore.ieee.org/document/7300596/
\item[Full text link:] http://ieeexplore.ieee.org/stamp/stamp.jsp?arnumber=7300596
\item[Comments:] This paper puts forward some interesting architecture for software localisation. It focuses on translating to Hndi, but could be transferable between many different languages.
\end{description}

\bibliographystyle{ieeetran}
\bibliography{initial_references}

\end{document}
