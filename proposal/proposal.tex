\documentclass{scrartcl}

\usepackage[hidelinks]{hyperref}
\usepackage[none]{hyphenat}

\title{Essay Proposal}
\subtitle{COMP160 - Software Engineering Essay}

\author{KG197307}

\begin{document}

\maketitle

\section*{Topic}

My essay will be on:
The economic impact of making code reusable and reusing code between computer platforms and environments including robotics.

\section*{Paper 1}
\begin{description}
\item[Title:] On device abstractions for portable, reusable robot code
\item[Citation:] \cite{RobotVaughan}
\item[Abstract:] ``We seek to make robot programming more efficient by developing a standard abstract interface for robot hardware, based on familiar techniques from operating systems and network engineering. This paper describes the application of three well known abstractions, the character device model, the interface/driver model, and the client/server model to this purpose. These abstractions underlie Player/Stage, our open source project for rapid development of robot control systems. One product of this project is the Player Abstract Device Interface (PADI) specification, which defines a set of interfaces that capture the functionality of logically similar sensors and actuators. This specification is the central abstraction that enables Player-based controllers to run unchanged on a variety of real and simulated devices. We propose that PADI could be a starting point for development of a standard platform for robot interfacing, independent of Player, to enable code portability and re-use, while still providing access to the unique capabilities of individual devices.''
\item[Web link:] \url{http://ieeexplore.ieee.org.ezproxy.falmouth.ac.uk/document/1249233/}
\item[Full text link:] \url{http://ieeexplore.ieee.org.ezproxy.falmouth.ac.uk/stamp/stamp.jsp?arnumber=1249233}
\item[Comments:] Interesting main point of the paper focuses on the goal of making a standard abstract interface for programming for robot hardware which covers many purposes using such a wide variety of hardware specs.
\end{description}

\section*{Paper 2}
\begin{description}
\item[Title:] The Scalability-Efficiency/Maintainability-Portability Trade-Off in Simulation Software Engineering: Examples and a Preliminary Systematic Literature Review
\item[Citation:] \cite{SimulationMehl}
\item[Abstract:] Large-scale simulations play a central role in science and the industry. Several challenges occur when building simulation software, because simulations require complex software developed in a dynamic construction process. That is why simulation software engineering (SSE) is emerging lately as a research focus. The dichotomous trade-off between scalability and efficiency (SE) on the one hand and maintainability and portability (MP) on the other hand is one of the core challenges. We report on the SE/MP trade-off in the context of an ongoing systematic literature review (SLR). After characterizing the issue of the SE/MP trade-off using two examples from our own research, we (1) review the 33 identified articles that assess the trade-off, (2) summarize the proposed solutions for the tradeoff, and (3) discuss the findings for SSE and future work. Overall, we see evidence for the SE/MP trade-off and first solution approaches. However, a strong empirical foundation has yet to be established; general quantitative metrics and methods supporting software developers in addressing the trade-off have to be developed. We foresee considerable future work in SSE across scientific communities.
\item[Web link:] http://ieeexplore.ieee.org.ezproxy.falmouth.ac.uk/document/7839468/
\item[Full text link:] http://ieeexplore.ieee.org.ezproxy.falmouth.ac.uk/stamp/stamp.jsp?arnumber=7839468
\item[Comments:] 
I found how the paper focuses on trying to balance Scalability and Efficiency "SE" with Maintainability and Portability "MP" to achieve high hardware efficiency. 
However to get this high hardware efficiency it states code needs tailoring to the specific hardware, therefore losing the maintainability and portability aspects. This links very well with my other research papers but on a larger and more broad scale. 
\end{description}

\section*{Paper 3}
\begin{description}
\item[Title:] Object-oriented and classical software engineering
\item[Citation:] \cite{ReusableSchach}
\item[Full text link:] https://worayoot.files.wordpress.com/2012/10/object-oriented-and-classical-software-engineering-8th-edition.pdf
\item[Comments:] Goes through basics of code reuse and obstacles programmers may come across. 
Explaining the benefits of portable code.
A broad source like this helps cover the basics.
\end{description}

\section*{Paper 4}
\begin{description}
\item[Title:] Software engineering with reusable components
\item[Citation:] \cite{ReusableSametinger}
\item[Abstract:] The book provides a clear understanding of what software reuse is, where the problems are, what benefits to expect, the activities, and its different forms. The reader is also given an overview of what sofware components are, different kinds of components and compositions, a taxonomy thereof, and examples of successful component reuse. An introduction to software engineering and software process models is also provided.
\item[Web link:] http://dl.acm.org/citation.cfm?id=1965498
\item[Full text link:] http://www.hostemostel.com/software/41.pdf
\item[Comments:] 
Links with Schach's book of code re-purposing covering requirements with code between interfaces and the economical aspects of reusing code and what the impacts are of making code reusable. 
Covering another aspect of my essay question while still relating to others.
\end{description}

\section*{Paper 5}
\begin{description}
\item[Title:] Title of paper
\item[Citation:] \cite{bibtex_key}
\item[Abstract:] Copy and paste the abstract here
\item[Web link:] Give the URL of the paper in IEEE Xplore, ACM Digital Library, or similar
\item[Full text link:] Give the URL of a downloadable PDF of the paper, if you can find one
\item[Comments:] Write a few sentences on how you found the article and why you believe it is relevant and/or important.
\end{description}

\bibliographystyle{ieeetran}
\bibliography{initial_references}

\end{document}
\section*{Paper 5}
\begin{description}
\item[Title:] Title of paper
\item[Citation:] \cite{bibtex_key}
\item[Abstract:] Copy and paste the abstract here
\item[Web link:] Give the URL of the paper in IEEE Xplore, ACM Digital Library, or similar
\item[Full text link:] Give the URL of a downloadable PDF of the paper, if you can find one
\item[Comments:] Write a few sentences on how you found the article and why you believe it is relevant and/or important.
\end{description}

\bibliographystyle{ieeetran}
\bibliography{initial_references}

\end{document}
\section*{Paper 5}
\begin{description}
\item[Title:] Title of paper
\item[Citation:] \cite{bibtex_key}
\item[Abstract:] Copy and paste the abstract here
\item[Web link:] Give the URL of the paper in IEEE Xplore, ACM Digital Library, or similar
\item[Full text link:] Give the URL of a downloadable PDF of the paper, if you can find one
\item[Comments:] Write a few sentences on how you found the article and why you believe it is relevant and/or important.
\end{description}

\bibliographystyle{ieeetran}
\bibliography{initial_references}

\end{document}
\section*{Paper 5}
\begin{description}
\item[Title:] Title of paper
\item[Citation:] \cite{bibtex_key}
\item[Abstract:] Copy and paste the abstract here
\item[Web link:] Give the URL of the paper in IEEE Xplore, ACM Digital Library, or similar
\item[Full text link:] Give the URL of a downloadable PDF of the paper, if you can find one
\item[Comments:] Write a few sentences on how you found the article and why you believe it is relevant and/or important.
\end{description}

\bibliographystyle{ieeetran}
\bibliography{initial_references}

\end{document}
\section*{Paper 5}
\begin{description}
\item[Title:] Title of paper
\item[Citation:] \cite{bibtex_key}
\item[Abstract:] Copy and paste the abstract here
\item[Web link:] Give the URL of the paper in IEEE Xplore, ACM Digital Library, or similar
\item[Full text link:] Give the URL of a downloadable PDF of the paper, if you can find one
\item[Comments:] Write a few sentences on how you found the article and why you believe it is relevant and/or important.
\end{description}

\bibliographystyle{ieeetran}
\bibliography{initial_references}

\end{document}
\section*{Paper 5}
\begin{description}
\item[Title:] Title of paper
\item[Citation:] \cite{bibtex_key}
\item[Abstract:] Copy and paste the abstract here
\item[Web link:] Give the URL of the paper in IEEE Xplore, ACM Digital Library, or similar
\item[Full text link:] Give the URL of a downloadable PDF of the paper, if you can find one
\item[Comments:] Write a few sentences on how you found the article and why you believe it is relevant and/or important.
\end{description}

\bibliographystyle{ieeetran}
\bibliography{initial_references}

\end{document}
\section*{Paper 5}
\begin{description}
\item[Title:] Title of paper
\item[Citation:] \cite{bibtex_key}
\item[Abstract:] Copy and paste the abstract here
\item[Web link:] Give the URL of the paper in IEEE Xplore, ACM Digital Library, or similar
\item[Full text link:] Give the URL of a downloadable PDF of the paper, if you can find one
\item[Comments:] Write a few sentences on how you found the article and why you believe it is relevant and/or important.
\end{description}

\bibliographystyle{ieeetran}
\bibliography{initial_references}

\end{document}
