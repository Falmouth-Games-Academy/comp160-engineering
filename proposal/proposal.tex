\documentclass{scrartcl}

\usepackage[hidelinks]{hyperref}
\usepackage[none]{hyphenat}

\title{Essay Proposal - Algorithms created for other purposes have been used to improve game development and architecture}

\subtitle{COMP160 - Computer Architecture Essay}

\author{1506919}

\begin{document}

\maketitle

\section*{Topic}

In my essay I would like to discuss whether, new specialist algorithms, that have been created outside the game programming field improve game development and architecture when applied to the game industry. When new algorithms are being discussed and created it actually refers to an algorithm that will be used to replace an older perhaps slower algorithm than an algorithm with a new purpose. As most game development software already have so many algorithms written in (e.g. collision, gravity etc.) there is no obvious need create and incorporate new specialist algorithms, which could mean those capable of writing original and new algorithms are becoming rare. Erin Jonathan Hastings, Ratan K. Guha and Kenneth O. Stanley use interactive evolutionary algorithms to generate content based on the behavior and interests of the player \cite{hastings2009automatic}, although they are using a the algorithm for a new purpose the algorithm was actually created in 1986 by R. Dawkins \cite{dawkins1986blind}, the algorithm has been used on 2-D/ 3-D art and music before being applied to game programming.


\section*{Paper 1}
\begin{description}
\item[Title:] Automatic Content Generation in the Galactic Arms Race Video Game
\item[Citation:] \cite{hastings2009automatic}
\item[Abstract:] Simulation and game content includes the levels, models, textures, items, and other objects encountered and possessed by players during the game. In most modern video games and in simulation software, the set of content shipped with the product is static and unchanging, or at best, randomized within a narrow set of parameters. However, ideally, if game content could be constantly and automatically renewed, players would remain engaged longer. This paper introduces two novel technologies that take steps toward achieving this ambition: 1) a new algorithm called content-generating NeuroEvolution of Augmenting Topologies (cgNEAT) is introduced that automatically generates graphical and game content while the game is played, based on the past preferences of the players, and 2) Galactic Arms Race (GAR), a multiplayer video game, is constructed to demonstrate automatic content generation in a real online gaming platform. In GAR, which is available to the public and playable online, players pilot space ships and fight enemies to acquire unique particle system weapons that are automatically evolved by the cgNEAT algorithm. A study of the behavior and results from over 1000 registered online players shows that cgNEAT indeed enables players to discover a wide variety of appealing content that is not only novel, but also based on and extended from previous content that they preferred in the past. Thus, GAR is the first demonstration of evolutionary content generation in an online multiplayer game. The implication is that with cgNEAT it is now possible to create applications that generate their own content to satisfy users, potentially reducing the cost of content creation and increasing entertainment value from single-player to massively multiplayer online games (MMOGs) with a constant stream of evolving content.
\item[Web link:] \url{http://ieeexplore.ieee.org.ezproxy.falmouth.ac.uk/document/5352259/?part=1}
\item[Full text link:] \url{http://ieeexplore.ieee.org.ezproxy.falmouth.ac.uk/document/5352259/?part=1}
\item[Comments:]  I found this article searching for new algorithms in game creation software and further proves my point that though the algorithm is used for a new purpose the algorithm itself is not new.
\end{description}

\section*{Paper 2}
\begin{description}
\item[Title:] A new method for generating 3-D face models for personalized user interaction
\item[Citation:] \cite{erdem2005new}
\item[Abstract:] A new method for generating and animating a 3-D model of a person's face is proposed. The method involves novel algorithms for 2-D to 3-D construction under perspective projection model, real-time mesh deformation using a lower-resolution control mesh, and texture image creation that involves texture blending in 3-D. The resulting face models can be readily used in 3-D games, mobile messaging, e-learning applications such as lip-reading and personalized training, 3-D movies and 3-D TV programs, and for providing human-computer interaction (HCI). User interfaces with personalized face models could be used to guide people in accessing knowledge and information on the Internet or to facilitate the usage of computers and software. Thus, it is believed that the utilization of personalized 3-D face models will help bring down the barrier between computers and people.
\item[Web link:] http://ieeexplore.ieee.org.ezproxy.falmouth.ac.uk/xpls/icp.jsp?arnumber=7078505
\item[Full text link:] http://ieeexplore.ieee.org.ezproxy.falmouth.ac.uk/xpls/icp.jsp?arnumber=7078505
\item[Comments:] Face models are a great example of re-purposing an algorithm and making it better before using it in the game industry, face models were being used in the 1970s but only for animations but are now being used in AAA games e.g. GTA 5.
\end{description}


\section*{Paper 3}
\begin{description}
\item[Title:] Performance-Driven Facial Animation 
\item[Citation:] \cite{williams1990performance}
\item[Abstract:] As computer graphics technique rises to the challenge of rendering lifelike performers, more lifelike performance is required. The techniques used to animate robots, arthropods, and suits of armor, have been extended to flexible surfaces of fur and flesh. Physical models of muscle and skin have been devised. But more complex databases and sophisticated physical modeling do not directly address the performance problem. The gestures and expressions of a human actor are not the solution to a dynamic system. This paper describes a means of acquiring the expressions of real faces, and applying them to computer-generated faces. Such an "electronic mask" offers a means for the traditional talents of actors to be flexibly incorporated in digital animations. Efforts in a similar spirit have resulted in servo-controlled "animatrons," high-technology puppets, and CG puppetry [1]. The manner in which the skills of actors and puppetteers as well as animators are accommodated in such systems may point the way for a more general incorporation of human nuance into our emerging computer media.The ensuing description is divided into two major subjects: the construction of a highly-resoved human head model with photographic texture mapping, and the concept demonstration of a system to animate this model by tracking and applying the expressions of a human performer.
\item[Web link:] http://dl.acm.org/citation.cfm?id=97906
\item[Full text link:] http://dl.acm.org/citation.cfm?id=97906
\item[Comments:] I found this paper referenced in paper 2 and I can use this as an example to show how an algorithm develops over time, each programmer changing the algorithm slowly improving the accuracy or simplicity.
\end{description}

\section*{Paper 4}
\begin{description}
\item[Title:] A novel application of inertial sensor based foot-mounted wearable electronic device
\item[Citation:] \cite{zhou2016novel}
\item[Abstract:] Wearable electronic devices have experienced an increasing development with the advances in the semiconductor industry and are receiving more attention during last decades. This paper presents a brand new application of inertial sensor-based foot-mounted wearable electronic device: game play. The introduced device is able to monitor and identify human foot stepping directions in real time, and coordinate these motions to control the player operation in games. This paper provides an overview of the previously-developed system platforms, introduces the main idea behind this novel application, and describes the implemented human foot moving direction identification algorithm. Practical experiment results demonstrate that the proposed system is capable of recognizing five foot motions: jump, step left, step right, step forward, and step backward, and has achieved a convincing performance for different users. The functionality of the system for real-time application has also been verified through the practical experiments.
\item[Web link:] http://ieeexplore.ieee.org.ezproxy.falmouth.ac.uk/document/7809956/
\item[Full text link:] http://ieeexplore.ieee.org.ezproxy.falmouth.ac.uk/document/7809956/
\item[Comments:] This paper gives implements an algorithm developed from software not originally created for games, I thought this might add some diversity in my paper on how algorithms can be used on new controllers.
\end{description}

\section*{Paper 5}
\begin{description}
\item[Title:] Design and implementation of high performance mobile game on embedded device
\item[Citation:] \cite{yang2010design}
\item[Abstract:] As mobile game grows rapidly in recent years, it becomes one of the new economic contributors to the game market. However the low-powered mobile device is the main barrier for the quick development of mobile game. Through mobile billiards game on the e868 mobile phone of the Bird-company, a game development framework for low performance mobile device is designed in this paper. To improve both the speed and visual effect of the game, this paper proposes an efficient collision detection algorithm suitable for mobile games from the aspect of game logic. Combined with the features of low-powered mobile devices, many different optimization techniques are presented and implemented.
\item[Web link:] http://ieeexplore.ieee.org.ezproxy.falmouth.ac.uk/document/5619347/
\item[Full text link:] http://ieeexplore.ieee.org.ezproxy.falmouth.ac.uk/xpls/icp.jsp?arnumber=5619347
\item[Comments:] This paper shows how algorithms have to be changed in order to work on different platforms, so in this example the author has to transfer a game that would work on computer and change the algorithm to make it playable on a phone
\end{description}

\bibliographystyle{ieeetran}
\bibliography{initial_references}

\end{document}
