\documentclass{scrartcl}

\usepackage[hidelinks]{hyperref}
\usepackage[none]{hyphenat}

\title{Essay Proposal}
\subtitle{COMP160 - Software Engineering Essay}

\author{Your Name Here}

\begin{document}

\maketitle

\section*{Topic}

My essay will be on:

% Add details as appropriate.

\section*{Paper 1}
% This is an example! Replace the details with a paper relevant to your chosen topic.
\begin{description}
\item[Title:] Game logic portability
\item[Citation:] \cite{GameLogic}
\item[Abstract:] ``Many game engines integrate the game logic with the graphics engine. In this paper we separate the two, thus making the logic portable between game engines. In our architecture the logic is represented as an ontology and a set of rules for a particular application domain. A mediator with an embedded rules-engine links the logic to a suitable game engine.We demonstrate our architecture in two ways. First, we show a traffic accident scenario running on two different game engines, with a separate mediator for each engine. The logic type is smart-terrain logic, with participants triggering events based on interaction and proximity tests. In the second demonstration (a simple first-person shooting game) we show the extensibility and performance of the architecture to control non-player characters quickly manoeuvring using proximity tests and waypoints.''
\item[Web link:] \url{http://dl.acm.org.ezproxy.falmouth.ac.uk/citation.cfm?id=1178580&CFID=744716189&CFTOKEN=30520581}
\item[Full text link:] \url{http://delivery.acm.org.ezproxy.falmouth.ac.uk/10.1145/1180000/1178580/p458-binsubaih.pdf?ip=193.61.64.8&id=1178580&acc=ACTIVE%20SERVICE&key=BF07A2EE685417C5%2EEAA225A8AB01C582%2E4D4702B0C3E38B35%2E4D4702B0C3E38B35&CFID=744716189&CFTOKEN=30520581&__acm__=1490818093_baf8c55578734d4e5306ddc7bad9d4cf}
\item[Comments:] This paper demonstrates logic portability between game engines.
\end{description}

\section*{Paper 2}
\begin{description}
\item[Title:] Digtal STB Game Portability Based on MVC Pattern
\item[Citation:] \cite{STB}
\item[Abstract:] Due to the tight coupling between the game and the STB environment, the ability to migrate the game to different STB environments is greatly restricted. We first analyzed the STB game software porting difficulties, and then put forward a solution to address the causes identified, presented an architecture based on Model-View-Controller pattern. The architecture managed to allow the same game to be ported to different STB environment without or with a little modifying the game. Lastly the architecture performance which revealed the architecture performance strengths and weaknesses was evaluated.
\item[Web link:] \url{http://ieeexplore.ieee.org.ezproxy.falmouth.ac.uk/document/5718337/}
\item[Full text link:] \url{http://ieeexplore.ieee.org.ezproxy.falmouth.ac.uk/stamp/stamp.jsp?arnumber=5718337}
\item[Comments:] This paper analyses STB game porting difficulties, puts forward a solution to address the causes identified and presents an architecture based on MVC.
\end{description}

\section*{Paper 3}
\begin{description}
\item[Title:] 
Improving program productivity, performance and portability through a high level language for graphics and game development
\item[Citation:] \cite{Geraci:2010:IPP:1836845.1836898}
\item[Abstract:] Our work focuses on the area of using a high level language to improve program productivity, performance and portability. In general, this has been an area of intense research. There are a number of previous efforts including ZPL [Chamberlain and et al 2004], X10/Fortress/Chapel from IBM/SUN/Cray [Weiland 2007], Intel's CT/RapidMind [McCool 2006] and parallel VSIPL++ [Lebak and et al 2005] to name a few. However, while these languages do great things in simplifying parallel implementation of code, extensions beyond that are limited. The primary exception to this is VSIPL++ which implements several high level functions useful to the signal processing community. While most of these languages can be used to implement graphics or game related algorithms if necessary, none of them attempt to provide a platform that makes such development particularly easy. On the other hand, high level engines such as Renderman and Unreal provide the wanted abstractions but with little or no guarantees about extensibility, portability, or parallel performance. Our research focuses on adapting the parallel VSIPL++ API from the signal processing community to the graphics and game development environment.
\item[Web link:] \url{http://dl.acm.org.ezproxy.falmouth.ac.uk/citation.cfm?id=1836898&CFID=744716189&CFTOKEN=30520581}
\item[Full text link:] \url{http://delivery.acm.org.ezproxy.falmouth.ac.uk/10.1145/1840000/1836898/a49-geraci.pdf?ip=193.61.64.8&id=1836898&acc=ACTIVE%20SERVICE&key=BF07A2EE685417C5%2EEAA225A8AB01C582%2E4D4702B0C3E38B35%2E4D4702B0C3E38B35&CFID=744716189&CFTOKEN=30520581&__acm__=1490818863_0b589eaf2ab80dbb16a0d87405947cb5}
\item[Comments:] This paper focuses on improving program productivity, performance and portability.
\end{description}

\section*{Paper 4}
\begin{description}
\item[Title:] Addressing openness and portability in outdoor pervasive role-playing games
\item[Citation:] \cite{6579529}
\item[Abstract:] This article introduces Barbarossa, an outdoor pervasive role-playing game. Existing pervasive game prototypes do not enable relocation of the game space as they heavily rely in orchestration actions. They also overlook several aspects which critically affect user acceptance and game experience such as scenario design, usability of employed technologies, game duration and intensity. Barbarossa addresses the abovementioned issues featuring portable game modes through moderating or seamlessly embedding any orchestration actions needed within the game process. It also takes into account concrete technology usage requirements for each game mode according to the game session duration and player effort required. Game experience is enhanced through incorporating several contextual parameters, while game rules may be personalized based on players' profile data retrieved from social networks.
\item[Web link:] \url{http://ieeexplore.ieee.org.ezproxy.falmouth.ac.uk/document/6579529/}
\item[Full text link:] \url{http://ieeexplore.ieee.org.ezproxy.falmouth.ac.uk/stamp/stamp.jsp?arnumber=6579529}
\item[Comments:] This article addresses openness and portability in an outdoor pervasive rpg. 
\end{description}

\section*{Paper 5}
\begin{description}
\item[Title:] Practices and Technologies in Computer Game Software Engineering
\item[Citation:] \cite{ComputerGameSE}
\item[Abstract:] Computer games are rich, complex, and often large-scale software applications. They're a significant, interesting, and often compelling domain for innovative research in software engineering techniques and technologies. Computer games are progressively changing the everyday world in many positive ways. Game developers, whether focusing on entertainment market opportunities or game-based applications in nonentertainment domains such as education, healthcare, defense, or scientific research (that is, serious games), thus share a common interest in how best to engineer game software. This article examines techniques and technologies that inform contemporary computer game software engineering.
\item[Web link:] \url{http://ieeexplore.ieee.org.ezproxy.falmouth.ac.uk/document/7819395/}
\item[Full text link:] \url{http://ieeexplore.ieee.org.ezproxy.falmouth.ac.uk/stamp/stamp.jsp?arnumber=7819395&tag=1}
\item[Comments:] This paper covers practices and technologies in computer game software engineering.
\end{description}

\section*{Paper 6}
\begin{description}
\item[Title:] Software portability: still an open issue?
\item[Citation:] \cite{Tanner:1996:SPS:234999.235001}
\item[Abstract:] Portability is widely regarded as a
done deal, but, although progress has
been made, the problem has not been
solved; if anything, it is becoming
more complex. The commercial
impact of non-portability increases as
information systems become more
distributed and interoperability
becomes a higher priority. This article
explores the issues behind the issues and their
technical and commercial impact. We also outline
some possible solutions being evaluated, particularly
within the X/Open community of IT buyers and
suppliers. Proposals such as a “what-works-withwhat”
information base and procurement assurance
mechanisms are explored.
\item[Web link:] \url{http://dl.acm.org.ezproxy.falmouth.ac.uk/citation.cfm?id=235001&CFID=744716189&CFTOKEN=30520581}
\item[Full text link:] \url{http://delivery.acm.org.ezproxy.falmouth.ac.uk/10.1145/240000/235001/p88-tanner.pdf?ip=193.61.64.8&id=235001&acc=ACTIVE%20SERVICE&key=BF07A2EE685417C5%2EEAA225A8AB01C582%2E4D4702B0C3E38B35%2E4D4702B0C3E38B35&CFID=744716189&CFTOKEN=30520581&__acm__=1490820932_4f47ccbf52083ecfc0e744046e68f95b}
\item[Comments:] This article discusses whether software portability is still an open issue.
\end{description}

\section*{Paper 7}
\begin{description}
\item[Title:] Tele-immersive Gaming for Everybody
\item[Citation:] \cite{Arefin:2011:TGE:2072298.2072455}
\item[Abstract:] In this demonstration, we present two 3D tele-immersive games:
light-saber dual and block fencing that merge 3D video representations
of participants in real-time to enable remote interactions in a
virtual world. The light-saber dual arranges participants in a symmetric
setup where both participants interact with each other in a
virtual world with similar goals. On the other hand, the block fencing
creates an asymmetric setup where participants interact with
virtual objects having different goals. Using these two setups, we
address the challenges and novelty of our solutions in portable environment
setup, data acquisition, multi-stream synchronization,
multi-stream session management, mobile device rendering, and
overlay communication in the design and implementation of advanced
3D tele-immersive systems.
\item[Web link:] \url{http://dl.acm.org.ezproxy.falmouth.ac.uk/citation.cfm?id=2072455&CFID=744716189&CFTOKEN=30520581}
\item[Full text link:] \url{http://delivery.acm.org.ezproxy.falmouth.ac.uk/10.1145/2080000/2072455/p783-arefin.pdf?ip=193.61.64.8&id=2072455&acc=ACTIVE%20SERVICE&key=BF07A2EE685417C5%2EEAA225A8AB01C582%2E4D4702B0C3E38B35%2E4D4702B0C3E38B35&CFID=744716189&CFTOKEN=30520581&__acm__=1490821372_d3eb55529f4f567e2f27ca8689514594}
\item[Comments:] This paper presents two 3D tele-imersive games and address the challenges and novelty of their solutions in a portable environment setup. 
setup
\end{description}

\section*{Paper 8}
\begin{description}
\item[Title:] Towards portable source code representations using XML
\item[Citation:] \cite{891464}
\item[Abstract:] One of the most important issues in source code analysis and software re-engineering is the representation of software code text at an abstraction level and form suitable for algorithmic processing. However, source code representation schemes must be compact, accessible by well defined application programming interfaces (APIs) and above all portable to different operating platforms and various CASE tools. This paper proposes a program representation technique that is based on language domain modes and the XML markup language. In this context, source code is represented as XML DOM trees that offer a higher level of openness and portability than custom-made tool specific abstract syntax trees. The DOM trees can be exchanged between tools in textual or binary form. Similarly the domain model allows for language entities to be associated with analysis services offered by various CASE tools, leading to an integrated software maintenance environment.
\item[Web link:] \url{http://ieeexplore.ieee.org.ezproxy.falmouth.ac.uk/document/891464/}
\item[Full text link:] \url{http://ieeexplore.ieee.org.ezproxy.falmouth.ac.uk/stamp/stamp.jsp?arnumber=891464}
\item[Comments:] This paper talks about portable source code representations using XML. 
\end{description}

\section*{Paper 9}
\begin{description}
\item[Title:] Comparative analysis of porting strategies in J2ME games
\item[Citation:] \cite{PortingJ2ME}
\item[Abstract:] Porting is a critical task in mobile device game development. The high diversity of devices requires correspondingly customized versions of a single game. Managing the variabilities among these customized versions while exploring the latent game commonality cannot be solely addressed with a single technique. This paper contributes to this task by identifying and analyzing porting challenges, and by evaluating and contrasting existing approaches within industrial-strength case studies of J2ME games. Further, we present lessons learned, proposing more effective guidelines for this process, aiming at improving the quality of the resulting applications and porting process productivity.
\item[Web link:] \url{http://ieeexplore.ieee.org.ezproxy.falmouth.ac.uk/document/1510109/}
\item[Full text link:] \url{http://ieeexplore.ieee.org.ezproxy.falmouth.ac.uk/stamp/stamp.jsp?arnumber=1510109}
\item[Comments:] This paper compares porting strategies in J2ME games and explains that porting is a critical task in mobile game development. 
\end{description}

\section*{Paper 10}
\begin{description}
\item[Title:] Porting mobile games in an aspect-oriented way: An industrial case study
\item[Citation:] \cite{6642506}
\item[Abstract:] Portability is a crucial requirement in the mobile game domain. Aspect-oriented programming has been shown to be a promising solution to implement the portability concerns, and more generally, to be a key technical enabler to transition mobile application development toward systematic software reuse. In this paper, we report a case study that critically examines how aspect orientation is practiced in industrial-strength mobile game applications. Our analysis takes into account technical artifacts, organizational structures, and their relationships. Altogether these complementary and synergistic viewpoints offer some concrete insights into developing information reuse and integration strategies in the rapidly changing landscape of mobile software development.
\item[Web link:] \url{http://ieeexplore.ieee.org.ezproxy.falmouth.ac.uk/document/6642506/}
\item[Full text link:] \url{http://ieeexplore.ieee.org.ezproxy.falmouth.ac.uk/stamp/stamp.jsp?arnumber=6642506}
\item[Comments:] This paper discusses porting mobile games in an aspec-oriented way. 
\end{description}

\bibliographystyle{IEEEtran}
\bibliography{initial_references}

\end{document}
