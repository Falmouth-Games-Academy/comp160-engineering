\documentclass{scrartcl}

\usepackage[hidelinks]{hyperref}
\usepackage[none]{hyphenat}

\title{Essay Proposal}
\subtitle{COMP160 - Software Engineering Essay}

\author{Michail Karakasis}

\begin{document}

\maketitle

\section*{Topic}

My essay will be on:

% Add details as appropriate.

\section*{Paper 1}
\begin{description}
\item[Title:] Video gaming accessibility
\item[Citation:] \cite{McPheron}
\item[Abstract:] ``Video games are becoming less and less usable by people with disabilities every day. These disabilities include cognitive, visual, and mobility. The purpose of this paper is to discuss the necessity of making video games accessible. This will be done through looking at benefits, current work for all disabilities, and how it's looked at by the law. The topic of video gaming accessibility is especially important in today's age with the disabled population rising. It is important to not let the disabled be left out of the enjoyment of video games.''
\item[Web link:] \url {http://ieeexplore.ieee.org/document/7272966/}
\item[Full text link:] \url {http://ieeexplore.ieee.org/stamp/stamp.jsp?arnumber=7272966}
\item[Comments:] 
\end{description}

\section*{Paper 2}
\begin{description}
\item[Title:] Guidelines of Serious Game Accessibility for the Disabled
\item[Citation:] \cite{HJ}
\item[Abstract:] ``Web accessibility/mobile app accessibility for the disabled has been studied for the past 10 years. The legislation of web accessibility guidelines can guarantee easy access to web contents of the disabled but not game playing, because web/ mobile app contents consists of rather simple information compared with game contents. Game contents include many PCs (player characters)/NPCs (non PCs) and the conflicts among them, as well as upgrade of a player character by completion of a quest in competition. Therefore, it is necessary to analyze and classify game accessibility so as to make the game accessibility guidelines.''
\item[Web link:] \url {http://ieeexplore.ieee.org/document/6579380/}
\item[Full text link:] \url {http://ieeexplore.ieee.org/stamp/stamp.jsp?arnumber=6579380}
\item[Comments:] 
\end{description}

\section*{Paper 3}
\begin{description}
\item[Title:] Designing Universally Accessible Games
\item[Citation:] \cite{Grammenos}
\item[Abstract:] ``Today, computer games are one of the major sources of entertainment. Computer games are usually far more demanding than typical interactive applications in terms of motor and sensory skills needed for interaction control, due to special-purpose input devices, complicated interaction techniques, and the primary emphasis on visual control and attention. This renders computer games inaccessible to a large percentage of people with disabilities. This article introduces the concept of universally accessible games, that is, games proactively designed to optimally fit and adapt to individual gamer characteristics and to be concurrently played among people with diverse abilities, without requiring particular adjustments or modifications. The concept is elaborated and tested through four case studies: a web-based chess game (UA-Chess), an action game (Access Invaders), a universally inaccessible game (Game Over!) used as an interactive educational tool, and an improved version of Access Invaders (Terrestrial Invaders). For all cases, key design and evaluation findings are discussed, reporting consolidated know-how and experience. Finally, the research challenge of creating multiplayer universally accessible games is further elaborated, proposing the novel concept of Parallel Game Universes as a potential solution.''
\item[Web link:] \url {http://dl.acm.org/citation.cfm?id=1486516}
\item[Full text link:] \url {https://www.ics.forth.gr/hci/files/selected_publications/2009_ACMCEM_DG-AS-CS.pdf}
\item[Comments:] 
\end{description}

\section*{Paper 4} 
\begin{description} 
\item[Title:] Towards generalised accessibility of computer games 
\item[Citation:] \cite{Arch} 
\item[Abstract:] ``Computer games accessibility have initially been regarded as an area of minor importance as there were much more “serious” topics to focus on. Today, the society is slowly moving forward in the direction of accessibility and the conditions come to make new proposals for mainstream game accessibility. In this paper we’ll show the main reasons why it is necessary to progress in this direction, then we’ll explain how works standard computer applications accessibility and why it is not working in general with games. We will discuss the state of the art in this area and finally we will introduce our vision of future accessibility framework allowing games developer to design accessible games as well as assistive providers the possibility of developing Assistive Games Interfaces.'' \item[Web link:] \url {https://link.springer.com/chapter/10.1007/978-3-540-69736-7_55} 
\item[Full text link:] \url {https://cedric.cnam.fr/fichiers/RC1513.pdf} 
\item[Comments:] 
\end{description}

\section*{Paper 5}
\begin{description}
\item[Title:] Evaluating and Investigating Game Accessibility for Deaf Players with the Semiotic Inspection Method
\item[Citation:] \cite{Flav}
\item[Abstract:] ``Semiotic Engineering is a Human Computer Interaction theory which perceives software as a communication between its designers and its users. We present here how an inspection method from that theory – the Semiotic Inspection Method - has been used in the context of action games to evaluate their accessibility. In particular, we present a case study in which we investigated how sound and music were employed to convey information to players and the impact on game experience from not being able to receive that information.''
\item[Web link:] \url {http://citeseerx.ist.psu.edu/viewdoc/summary;jsessionid=E1312F1230D976BE51AF70BBA7267F80?doi=10.1.1.261.4160}
\item[Full text link:] \url {http://gur.hcigames.com/wp-content/uploads/2015/02/Evaluating-and-Investigating-Game-Accessibility-for-Deaf-Players-with-the-Semiotic-Inspection-Method.pdf}
\item[Comments:]
\end{description}

\section*{Paper 6}
\begin{description}
\item[Title:] Game not over: Accessibility issues in video games
\item[Citation:] \cite{Bierre}
\item[Abstract:] ``An issue that has been facing the game industry recently is the need to provide accessible games. There are various legal, financial, and ethical reasons for wanting more accessible games. This paper will examine the scope of the problem by reviewing the need for accessibility, the current state of the industry, and some proposed initiatives that we feel should start to occur in the near future. We also will look at case studies of several commercial games that have provided accessibility features.''
\item[Web link:] \url {https://www.researchgate.net/publication/267403944_Game_Not_Over_Accessibility_Issues_in_Video_Games}
\item[Full text link:]
\item[Comments:]
\end{description}

\section*{Paper 7}
\begin{description}
\item[Title:] Game Accessibility: A Survey
\item[Citation:] \cite{Yuan}
\item[Abstract:] ``Over the last three decades, video games have evolved from a pastime into a force of change that is transforming the way people perceive, learn about, and interact with the world around them. In addition to entertainment, games are increasingly used for other purposes such as education or health. Despite this increased interest, a significant number of people encounter barriers when playing games due to a disability. Accessibility problems may include the following: (1) not being able to receive feedback; (2) not being able to determine in-game responses; (3) not being able to provide input using conventional input devices. This paper surveys the current state-of-the-art in research and practice in the accessibility of video games and points out relevant areas for future research. A generalized game interaction model shows how a disability affects ones ability to play games. Estimates are provided on the total number of people in the United States whose ability to play games is affected by a disability. A large number of accessible games are surveyed for different types of impairments, across several game genres, from which a number of high- and low-level accessibility strategies are distilled for game developers to inform their design.''
\item[Web link:] \url {http://dl.acm.org/citation.cfm?id=1966675}
\item[Full text link:] \url {https://www.cse.unr.edu/~fredh/papers/journal/29-gaas/paper.pdf}
\item[Comments:]
\end{description}

\section*{Paper 8}
\begin{description}
\item[Title:] Playmancer: Games for health with accessibility in mind
\item[Citation:] \cite{Kala}
\item[Abstract:] ``The term Serious Games has been used to describe computer and video games used as educational technology or as a vehicle for presenting or promoting a point of view. Serious games can be of any genre and many of them can be considered a kind of edutainment. Serious games are intended to provide an engaging, self-reinforcing context in which to motivate and educate the players towards knowledgeable processes, including business operations, training, marketing and advertisement. Serious games can be compelling, educative, provocative, disruptive and inspirational. The potential of games for entertainment and learning has been demonstrated thoroughly from both research and market. Unfortunately, the investments committed to entertainment dwarf what is committed for more serious purposes. In this feature, we will argue that the motives, incentives and expectations of the computer game industry differ from one cultural and economic environment to another. As the game industry is dominated by US companies, computer game products are targeting user groups mostly informed by the marketing departments of those companies. This process creates marginalised user groups and game types that are not addressed effectively by the computer game market. Accessible games and games for health comprise this underdeveloped niche. Research project PlayMancer is a multi-partner effort to tackle both of those issues in a coherent way.''
\item[Web link:] \url {https://www.researchgate.net/publication/46532710_PlayMancer_Games_for_Health_with_Accessibility_in_Mind}
\item[Full text link:] 
\item[Comments:] 
\end{description}

\section*{Paper 9}
\begin{description}
\item[Title:] Analyzing the Use of Sounds in FPS games and its Impact for Hearing Impaired Users
\item[Citation:] \cite{Denise}
\item[Abstract:] ``Although very popular, most FPS games are disliked by the Brazilian deaf community due to some difficulties faced, mainly with fast reactions that the player must have to proceed with the missions. FPS game’s designers use sound strategies to convey information to the player, and if the game does not provide some kind of visual redundancy, deaf or hard of hearing players encounter a bad game experience. Continuing a previous research, we performed a semiotic inspection in three games of the genre aforementioned (Half-Life 2, Modern Warfare 2, XIII). This paper corroborates the previously identified strategies that are used to provide information to players via audio and analyze how the lack of this information can impact a player who cannot hear, by identifying the redundancies provided by each of the three games.''
\item[Web link:] \url {http://citeseerx.ist.psu.edu/viewdoc/summary?doi=10.1.1.278.7996}
\item[Full text link:] \url {http://homepages.dcc.ufmg.br/~chaimo/public/SBGames12-denise.pdf}
\item[Comments:] 
\end{description}

\section*{Paper 10}
\begin{description}
\item[Title:] Teaching Programming to the Deaf
\item[Citation:] \cite{Ross}
\item[Abstract:] ``For certain groups of handicapped persons the field of computer science offers challenging, high-paying careers which are more accessible than other careers. Programming can be done at cathode ray tube terminals which provide a dynamic, visual environment needed by the hearing impaired, and which can be placed in locations convenient to those with mobility impairments. Unfortunately, most colleges and universities are not prepared to teach such students, particularly the deaf. In this paper an ongoing research project to design an online, dynamic, student-controlled library of programming language examples for use in teaching programming languages is described. This system promises to be useful in meeting the needs of handicapped students.''
\item[Web link:] \url {http://dl.acm.org/citation.cfm?id=964174}
\item[Full text link:] \url {http://www.catea.gatech.edu/scitrain/kb/FullText_Articles/Ross.pdf}
\item[Comments:]
\end{description}

\section*{Paper 11}
\begin{description}
\item[Title:] Evaluation of Software Tools with Deaf Children
\item[Citation:] \cite{Mich}
\item[Abstract:] ``Evaluating software applications with deaf or hard of hearing children requires methods and procedures tuned to them. Indeed, they are unusual users with special communication needs. This paper proposes a list of guidelines for organizing effective evaluations of interactive tools with deaf children. The novelty of this work is that such guidelines are not based on theoretical thinking. Instead, they are built on data collected through questionnaires proposed to experts working with deaf children. The questionnaire's data are reinforced by my experience which was gained during usability tests with deaf children. In future work, the effectiveness of these guidelines will be checked during the evaluation of an e-learning tool for Italian deaf children.''
\item[Web link:] \url {http://dl.acm.org/citation.cfm?id=1639692&preflayout=tabs}
\item[Full text link:] \url {https://i3.fbk.eu/files/assetts2009.pdf}
\item[Comments:]
\end{description}

\bibliographystyle{ieeetran}
\bibliography{initial_references}

\end{document}
