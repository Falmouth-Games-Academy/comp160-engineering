\documentclass{scrartcl}

\usepackage[hidelinks]{hyperref}
\usepackage[none]{hyphenat}

\title{Essay Proposal}
\subtitle{COMP160 - Software Engineering Essay}

\author{Your Name Here}

\begin{document}

\maketitle

\section*{Topic}

My essay will be on:
This paper will explore the Software available for digital game development to tackle issues related to accessibility, and to draw comparisons between digital Games and Web development. The requirements for websites to be considered accessible is well defined and there are many different documents across the globe detailing the exact specifications of what an website needs to be considered accessible. This is  the level of detail that this paper will be exploring, to see if the digital games industry has been meeting what web development has been calling the acceptable standard, and what documentation and software can be used to help aid in the creation of digital games to become more accessible to all potential users.

\section*{Paper 1}
\begin{description}
\item[Title:] Designing Software that is Accessible to Individuals with Disabilities and Making it More Useable by Everyone
\item[Citation:] \cite{Sheryl Burgstahler}
\item[Web link:] \url{https://www.washington.edu/doit/sites/default/files/atoms/files/Designing-Software-Accessible-Individuals-Disabilites.pdf}
\item[Full text link:] \url{https://www.washington.edu/doit/sites/default/files/atoms/files/Designing-Software-Accessible-Individuals-Disabilites.pdf}
\end{description}

\section*{Paper 2}
\begin{description}
\item[Title:] Analysis of UK Parliament Web Sites for Disability Accessibility 
\item[Citation:] \cite{Joanne M. Kuzma and Colin Price }
\item[Abstract:] "The growth of the Internet has led to an increase in the number of public services offered
by U.K. government entities on their Web sites. A variety of consumers use e-government sites, and
those individuals with disabilities are guaranteed the same access government sites under the U.K.’s
Disability Discrimination Act (DDA) of 1995. This law provides equality in access, and implements
penalties for non-adherence to the law. Industry standards also exist which helps site developers to
create better site accessibility. However, despite both standards and legal regulations, total openness
of sites for people with disabilities is still not widespread. The purpose of this study is to examine the
level of accessibility of a randomly selected sample of 130 members of the U.K. House of Commons.
Each site was analyzed using an online software tool –Truwex - to determine if they met industry Web
Content Accessibility Guidelines (WCAG) levels 1.0 and 2.0 standards and DDA law. The results
showed that the majority of the sites did not meet either guidelines or legal mandates. Many of the
sites displayed similar precedents when it came to the types of non-compliance, and could easily
improve compliance with minor changes." 
\item[Web link:] https://eprints.worc.ac.uk/617/1/kuzmaukdisaforconferencerevised.pdf 
\item[Full text link:] https://eprints.worc.ac.uk/617/1/kuzmaukdisaforconferencerevised.pdf
\end{description}

\section*{Paper 3}
\begin{description}
\item[Title:] Promoting inclusive design practice at the Global Game Jam: A pilot evaluation 
\item[Citation:] \cite{Michael}
\item[Abstract:] "Games are a popular form of entertainment. However, many computer games present unnecessary barriers to players with sensory, motor and cognitive impairments. In order to overcome such pitfalls, an awareness of their impact and a willingness to apply inclusive design practice is often necessary. The Global Game Jam offers a potential avenue to promote inclusive design practices to students of game development. As such, this paper evaluates the impact of an initiative to promote inclusive design practices during the 2014 Global Game Jam. An attitude questionnaire was distributed to both participants and non-participants at one event venue. The results indicate that, having enrolled in the initiative, students' attitudes improved. Furthermore, all attendees reported they were likely to pursue further learning opportunities and consider accessibility issues in their future games. This suggests that the Global Game Jam, and other similar events, present an attractive avenue to promote inclusive design practice within the context of digital game development. However, further analysis of submitted games, additional qualitative inquiry and a large-scale trial are needed to determine impact on practice and to form recommendations for future events." 
\item[Web link:] http://ieeexplore.ieee.org.ezproxy.falmouth.ac.uk/document/7044162/ 
\item[Full text link:] http://ieeexplore.ieee.org.ezproxy.falmouth.ac.uk/document/7044162/
\end{description}

\bibliographystyle{ieeetran}
\bibliography{initial_references}

\end{document}
