\documentclass{scrartcl}

\usepackage[hidelinks]{hyperref}
\usepackage[none]{hyphenat}

\title{Essay Proposal}
\subtitle{COMP160 - Software Engineering Essay}

\author{Phil Sparkes}

\begin{document}

\maketitle

\section*{Topic}

What main practices should a programmer follow in order to develop a game using an engine so that it can comfortably be ported to another platform?

% Add details as appropriate.

\section*{Paper 1}
% This is an example! Replace the details with a paper relevant to your chosen topic.
\begin{description}
\item[Title:] Enhancing Software Portability with a Testing and Evaluation Platform
\item[Citation:] \cite{Weisshardt}
\item[Abstract:] ``Recently a variety of service robots is available as standard platforms allowing a worldwide exchange of software for applications in the service sector or industrial environments. Open source software components enhance this sharing process, but require the maintenance of a certain level of quality. This paper presents an approach to a testing and evaluation platform which facilitates the sharing of capabilities over different robot types, environments and application cases.''
\item[Web link:] \url{http://ieeexplore.ieee.org.ezproxy.falmouth.ac.uk/document/6840132/}
\item[Full text link:] \url{http://ieeexplore.ieee.org.ezproxy.falmouth.ac.uk/stamp/stamp.jsp?arnumber=6840132}
\item[Comments:] 
\end{description}

\section*{Paper 2}
\begin{description}
\item[Title:] Software reuse in robotics: Enabling portability in the face of diversity
\item[Citation:] \cite{Smith}
\item[Abstract:]`` Software development for robotics applications is characterised by a high degree of specialisation. The reasons for this may centre on the diversity of robotic hardware, limitations on performance, and the need to perform complex and diverse tasks. The result of using such specialised software is an almost non-existent level or software portability. It is proposed that the use of abstraction can enable the use or component software and bring with it the benefits or reuse. Abstraction of robotic hardware and software is difficult, and it is clear that a single robot abstraction is not practical due to the degree or diversity. It is proposed that some middle ground between specialisation and complete abstraction can be found. A second level component framework using fuzzy logic techniques is presented to illustrate how a significant degree of abstraction can be achieved, facilitating software portability, while accommodating the diversity of robotics.''
\item[Web link:]\url{http://ieeexplore.ieee.org.ezproxy.falmouth.ac.uk/document/1438043/}
\item[Full text link:] \url{http://ieeexplore.ieee.org.ezproxy.falmouth.ac.uk/stamp/stamp.jsp?arnumber=1438043}
\item[Comments:] 
\end{description}

\section*{Paper 3}
\begin{description}
\item[Title:] Test software design techniques for reuse and portability
\item[Citation:] \cite{DeAbren}
\item[Abstract:] ``A test station software, developed using LabWindows/sup TM//CVI/sup TM/, was implemented using a flexible architecture and modular design techniques in order to facilitate reuse and portability during the various stages of product development. The various software components-User Interface, Instrument Control, Data Collection, and Results Analysis-were partitioned into independent sub-units with limited end clearly defined interfaces between each other. This eliminated the coupling of software modules across software components. A test station configuration software component was also designed to centralize access and distribution of test station data.''
\item[Web link:] \url{http://ieeexplore.ieee.org.ezproxy.falmouth.ac.uk/document/885610/}
\item[Full text link:]  \url{http://ieeexplore.ieee.org.ezproxy.falmouth.ac.uk/stamp/stamp.jsp?arnumber=885610}
\item[Comments:]
\end{description}

\section*{Paper 4}
\begin{description}
\item[Title:]Portability of interactive graphics software
\item[Citation:] \cite{Brittain}
\item[Abstract:] ``One solution to obtaining a portable graphics architecture is presented. By abstracting the functionality present in most 3-D graphics systems and augmenting it with advanced rendering features, a highly portable, efficient, and modern graphics architecture for interactive 3-D graphics applications (including modeling, animation, and scientific visualization) is obtained. Using appropriate object-oriented design procedures ensures the efficiency, maintainability, and portability of the architecture. The design and implementation of the graphics system used to achieve this high degree of portability are described.''
\item[Web link:] \url{http://ieeexplore.ieee.org.ezproxy.falmouth.ac.uk/document/56301/}
\item[Full text link:]  \url{http://ieeexplore.ieee.org.ezproxy.falmouth.ac.uk/stamp/stamp.jsp?arnumber=56301}
\item[Comments:]
\end{description}


\section*{Paper 5}
\begin{description}
\item[Title:] Portable software for multiprocessor systems
\item[Citation:] \cite{Colbrook}
\item[Abstract:] ``The authors describe Prelude, a programming language and accompanying system support for writing portable parallel programs for multiprocessor architectures. Prelude allows the programmer to separate the description of the computation to be performed by a program from the description of how that computation is to be mapped onto a machine. This makes it easier to tune the performance of a program on a particular machine and also simplifies porting a program to new architectures.''
\item[Web link:] \url{http://ieeexplore.ieee.org.ezproxy.falmouth.ac.uk/document/180499/}
\item[Full text link:]  \url{http://ieeexplore.ieee.org.ezproxy.falmouth.ac.uk/stamp/stamp.jsp?arnumber=180499}
\item[Comments:]
\end{description}

\section*{Paper 6}
\begin{description}
\item[Title:] High performance C programming
\item[Citation:] \cite{Myalapalli}
\item[Abstract:] ``Optimization is one of the desired objectives in software engineering to ensure that program makes proficient utilization of system resources. Optimization entails making program(s) bug free, reduced time and space complexity, portable etc. As such in this paper we present sundry tuning techniques for rewriting a statement which increases the efficiency of the statement in the area of time and space complexity. The denouements of our schemes and methodologies advocate that security and performance i.e. time and space complexities are enhanced. Our analysis can serve as hand held Tuning and White Box Testing Tool for C programmers.''
\item[Web link:] \url{http://ieeexplore.ieee.org.ezproxy.falmouth.ac.uk/document/7087005/}
\item[Full text link:]  \url{http://ieeexplore.ieee.org.ezproxy.falmouth.ac.uk/stamp/stamp.jsp?arnumber=7087005}
\item[Comments:]
\end{description}

\section*{Paper 7}
\begin{description}
\item[Title:] Technology and design tools for portable software development for embedded systems
\item[Citation:] \cite{Sedov}
\item[Abstract:] ``Nowadays embedded systems are used in broad range of domains such as avionics, space industry, automotive, mobile devices, domestic appliances and so on. There is enormous number of tasks that should be solved using embedded systems. There are many tools and approaches that allow developing of software for domain area experts, but mainly for general purpose computing systems. In this article we propose the complex technology and tools that allows involving domain experts in software development for embedded systems. The proposed technology has various aspects and abilities that can be used to build verifiable and portable software for a wide range of embedded platforms.''
\item[Web link:] \url{http://ieeexplore.ieee.org.ezproxy.falmouth.ac.uk/document/7000937/}
\item[Full text link:]  \url{http://ieeexplore.ieee.org.ezproxy.falmouth.ac.uk/stamp/stamp.jsp?arnumber=7000937}
\item[Comments:]
\end{description}

\section*{Paper 8}
\begin{description}
\item[Title:] On the GPU-CPU Performance Portability of OpenCL for 3D Stencil Computations
\item[Citation:] \cite{Su}
\item[Abstract:] ``Although OpenCL programming provides full code portability between different hardware platforms, performance portability can be far from satisfactory. In this work, we use a set of representative 3D stencil computations to study OpenCL's performance portability between GPUs and CPUs. For each stencil computation, we have devised different implementations of the computational kernel function, all being 100 code-portable between the two architectures. The most straightforward and compact implementation gives satisfactory CPU performance but performs poorly on GPUs, because such an implementation hampers effective use of the GPU hardware. By injecting code complexity into the involved loop nests, we can create kernel functions that still have full code portability but with increased performance portability. It is found that spatial data blocking and register reuse can be beneficial for performance on both GPUs and CPUs, whereas use of OpenCL's local memory (and subsequent temporal blocking) may only have positive effects on GPUs.''
\item[Web link:] \url{http://ieeexplore.ieee.org.ezproxy.falmouth.ac.uk/document/6808160/}
\item[Full text link:]  \url{http://ieeexplore.ieee.org.ezproxy.falmouth.ac.uk/stamp/stamp.jsp?arnumber=6808160}
\item[Comments:]
\end{description}

\section*{Paper 9}
\begin{description}
\item[Title:] Specification and-production techniques for portable software components
\item[Citation:] \cite{Hardy}
\item[Abstract:] ``.''
\item[Web link:] \url{http://ieeexplore.ieee.org.ezproxy.falmouth.ac.uk/document/766635/}
\item[Full text link:]  \url{http://ieeexplore.ieee.org.ezproxy.falmouth.ac.uk/stamp/stamp.jsp?tp=&arnumber=766635}
\item[Comments:]
\end{description}

\section*{Paper 10}
\begin{description}
\item[Title:] Why Johnny can't build [portable scientific software]
\item[Citation:] \cite{Dubois}
\item[Abstract:] ``The title of this article refers to Rudolph Flesch's famous 1955 book, "Why Johnny Can't Read", which called attention to a nationwide decline in reading ability. Here, the author wants to talk about another situation in which an important ability is lacking: the ability to create significant, portable scientific software. The author discusses some of the reasons this problem exists and suggests some approaches to solving it that seem promising.''
\item[Web link:] \url{http://ieeexplore.ieee.org.ezproxy.falmouth.ac.uk/document/1225867/}
\item[Full text link:]  \url{http://ieeexplore.ieee.org.ezproxy.falmouth.ac.uk/stamp/stamp.jsp?arnumber=1225867}
\item[Comments:]
\end{description}


\bibliographystyle{ieeetran}
\bibliography{initial_references}

\end{document}
