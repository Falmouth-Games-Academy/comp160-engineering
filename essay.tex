% Please do not change the document class
\documentclass{scrartcl}

% Please do not change these packages
\usepackage[hidelinks]{hyperref}
\usepackage[none]{hyphenat}
\usepackage{setspace}
\doublespace

% You may add additional packages here
\usepackage{amsmath}

% Please include a clear, concise, and descriptive title
\title{How does the engineering workflow of a video game differ from that of other software when trying to meet a particular Internationalisation standard?}

% Please do not change the subtitle
\subtitle{COMP160 - Software Engineering Essay}

% Please put your student number in the author field
\author{1606119}

\begin{document}

\maketitle

\abstract{Localisation of software has the capacity to pose many a challenge to developers, as there are so many factors and small details that come into play, with a large portions of the software needing to be completely overhauled depending on the locale, a set of "Internationalisation Standards"\cite{hogan2004key} have to be followed by the developer if they want their software to be understandable and usable in the local they are adapting it for, this is applicable to both video games and other types of software and so my essay will compare and contrast the different steps that the developers of these types of software have to take.}

\section{Introduction}

% Write your introduction here. A brief introduction is recommended, which should outline key details of the chosen topic and the reviewed papers, motivate the work, and provide a roadmap of key points to the reader. The motivation is quite important here, as essays should have a contribution (i.e., what is the point of the essay, and what does the reader take away from the essay) and the link between the motivation (in the introduction) and the contribution (in the conclusion) should be made clear.

As technology's accessibility and uptake across the globe is forever increasing, many new emerging markets opportunities are opening up for software developers and having localised versions of their software allows for ``Local users better understand and use it, attract more users, and increase software sales''\cite{6601827}. As previously mentioned, developers have to work to a set of standards to make sure that their software is understandable to a variety of markets, and through my research I have found that, due to the differences in the engineering workflow of video games and other types of software, that different organisations and groups have created several different sets of standards for different types of software, and as games require a lot more context and speech to relay the story to the player, the workflow is a lot larger, however there are some points where they overlap and flow in a similar way. 

\section{Integral Differences}
As I have found through my research, compared to other types of software, video games have both a much higher amount of content, and require much more work to localise than other types of software e.g. Buisness, as ``The linguistic port is more than simple translation; it involves use of vernacular and dialect as well as culturally appropriate interpretation of the game context'' \cite{losavio2014linguistic}, meaning the translation and it's implentation will need to use local language and the correct use of this to be fully understood; an increasing step that developers are taking is to ``Cultralize'' their software\cite{bestpractices}, which goes further into the games content and removes anything that may be seen as offensive or not well recieved in a particular location, this has become a standard of games internationalisation, as without the correct content being removed or replaced, it will not pass certification in that region, which extends the development time and wastes money and resources, while buisness software won't have any of these issues it doesn't need to be certification. Which when focusing on the modularity (Engineering Principle) of their software, game developers will need to focus on making sure that ``Resource files should store all text used in the game'' and that they ``do not hardcode text strings in the source code.''\cite{bestpractices}
%SIGHT THE PAPER WITH THIS IN   ,DRIVES GAME DEVS TO MAKE TEXT AND SPEECH MORE MODULAR, MORE TIME, may shortcut with star wars example!!

%as games are more complex and the story may have fictional elements, can't simply be translated, unlike buisness software which can be localised with a simple tool, cite paper about this, drive developers to make speech and story more modular (standard) in game as easier to replace, not ``hardcoded''
% unicode allows for easier translation


\section{Conclusion}

Write your conclusion here. The conclusion should do more than summarise the essay, making clear the contribution of the work and highlighting key points, limitations, and outstanding questions. It should not introduce any new content or information.

\bibliographystyle{ieeetran}
\bibliography{references}

\end{document}
