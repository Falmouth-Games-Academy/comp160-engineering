% Please do not change the document class
\documentclass{scrartcl}

% Please do not change these packages
\usepackage[hidelinks]{hyperref}
\usepackage[none]{hyphenat}
\usepackage{setspace}
\doublespace

% You may add additional packages here
\usepackage{amsmath}

% Please include a clear, concise, and descriptive title
\title{How can action games accommodate for hearing impaired gamers regarding conveying information?}

% Please do not change the subtitle
\subtitle{COMP160 - Software Engineering Essay}

% Please put your student number in the author field
\author{1607934}

\begin{document}

\maketitle

\abstract
Video games have been regarded as a major source of entertainment for countless of people. However, it has proven difficult for impaired individuals of varying kinds to enjoy or even play said games. As a result, this paper will analyse and address the various methods that can be implemented in action games to accomodate for hearing impaired gamers, both to improve their experience and generally provide more accessibility for a wider range of players. This will be done with aid from previous studies and by discussing potential alternatives, as well as how they can be implemented.

\section{Introduction}
It is easy to forget the importance of adapting games into meeting specific needs. While the majority of current games' visual and auditory qualities have improved over time, the same cannot be said for their accessibilities. Game accessibility involves creating features and strategies that aid impaired individuals' game experience, being able to truly play games even when under limiting conditions \cite{11}. Indeed, older existing games such as Half Life 2 \cite{Denise} and Mortal Kombat provide all kinds of visual and auditory feedback. As a result, one would think that current games would have expanded on them and increased the amount of accessibility options. Unfortunately, this is not the case for the majority. Therefore, this paper aims to analyse and discuss the importance of accessibility as well as relevant strategies from previous research regarding the hearing impaired, such as interaction feedback, communicating threat, etc. \cite{Denise}, which will feed into the conclusion of the research question. Additionally, this paper will focus primarily on action games, and explain the varying options they can provide to aid hearing impaired players in having an enjoyable and immersive experience.

\section{How important is accessibility?}
Simply making games `usable' for the sake of accessibility is not enough, as they have to be both enjoyable and able to provide the same level of playability that able-bodied players receive. Indeed, one can easily measure the amount of time a player needs and uses in a level, or see if the player is able to successfully complete said level. This is not enough, as games have to evoke varying kinds of emotions, something that hearing impaired players can easily miss by, for instance, failing to follow cutscenes that contain crucial plot information conveyed through speech, ultimately leaving them with disappointing and unsatisfactory feelings. Strong emotions influence people by allowing them to form their own responses and involve in critical thinking of solutions to problems as if they were of utmost importance. This forms a strong interactive bond between the player and the game - or even between two players, such as in multiplayer games. Being deprived of such emotionally rich content can lead to social interaction issues and could possibly affect cognitive development. 

\section{Implementation methods}
Various classes that exist in games such as Half Life 2, among others, have been discussed. These represent what features of the sound effects were used to convey information, such as \textbf{timbre}, \textbf{speech}, \textbf{sound volume} and their complementary strategies which involve \textbf{interaction feedback} (informing the player that an action was completed), \textbf{communicating threat} (informing of nearby threats), \textbf{characterization of game elements} (distinguishable sound sources), including many more \cite{Flav} \cite{Denise}. These can be used to evaluate a game's auditory framework and can serve as basic guidelines for game developers to determine how they can make their games as accessible as possible for the hearing impaired. For instance, regarding \textbf{interaction feedback} in Half Life 2, the player understands that various events, such as bullet impacts or enemy movements are occuring through the \textbf{timbre}, which helps the player distinguish between the two events, allowing them to perceive and understand what is happening around them. \textbf{Communicating threat} is another notable strategy. A relevant example taken from Call of Duty: Modern Warfare 2, where the player is informed of an imminent grenade explosion by the use of \textbf{speech} and \textbf{redundancies}, both by showing them a grenade icon indicating its location and character dialogue yelling at the player to  ``\textit{Get down!}'', although the latter not being as reliable for the hearing impaired. \cite{Denise}


Your essay must make a clear recommendation, in terms of which of the three techniques you have reviewed is the best according to whichever metric or metrics you feel is most appropriate. You must justify your choice, backing it up with empirical evidence. However remember that an academic essay is not a murder mystery: you should already have briefly discussed your recommendation in the introduction and in other parts of the essay. Do not save it for a grand reveal at the end.

\section{Recommendations}

* Audio cues

* Subtitles or closed captions

\section{Conclusion (wip)}

While there are a few newer games with accessibility features, it is not nearly enough - both for mainstream and non-mainstream games. 

\bibliographystyle{ieeetran}
\bibliography{references}

\end{document}

