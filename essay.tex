% Please do not change the document class
\documentclass{scrartcl}

% Please do not change these packages
\usepackage[hidelinks]{hyperref}
\usepackage[none]{hyphenat}
\usepackage{setspace}
\doublespace

% You may add additional packages here
\usepackage{amsmath}

% Please include a clear, concise, and descriptive title
\title{Engineering for Portability - The Portability of Games to Mobile Platforms}

% Please do not change the subtitle
\subtitle{COMP160 - Software Engineering Essay}

% Please put your student number in the author field
\author{1605708}

\begin{document}

\maketitle

\abstract{Mobile phones are no common place within both homes and people’s pockets. Many people use them for work, games and communication. As such, many programs that have been created for these purposes on PC are being imported to mobile platforms. The mobile phone industry having such a large concentration on casual games means many games have been ported from PC and console versions to mobile. This essay aims to look at the problems, practices and general focus that companies have when they port a game from other consoles to mobile. Overall I wish to find out "How has the mobile games industry influenced professional developers and their software practices, especially with regard to mobile games development.}}

\section{Introduction}
How has the development in demand for mobile games affected the choices in which platforms developers make games on/port games to?
Mobile gaming has become a huge part of the gaming market. Globally smart phone gaming revenue accounts for €12.1 billion each year, in the US 67 percent of adults play games. So it is fair to say that mobile devices have become a large part of how people play and pay for games. With this in mind how have developers made changes to their practices to work with this thriving industry? What choices do they make when they make software for mobile platforms? What is the focus when they port games from one platform to mobile platforms. What difficulties are there in making/porting software and games to mobile platforms? Overall asking, what can future developers who wish to make/port software for mobile platforms learn from what is already known?


\section{}
\cite{mobileAgility}
Agile, the commonly used method for software development, is very applicable to development for mobile platforms. However, there are several key changes that should be taken into account when developing for mobile instead of other platforms.
 Agility – using agile methodology, with a specific focus on the iterative and incremental process and test driven development. 
Market consciousness – having a focus on development as a business rather than as a project. Aware of other similar products already out there and pricing.
Software Product Line Support – Developing software with specific features and core assets in a set way that can be used across multiple systems. 
Architecture Based Development – Developing an architecture specific to the software’s needs. 
Support for reusability – Using familiar features such as text boxes that users will recognise easily. 
Inclusion of Review and Learning Sessions – Analysis of the product to ensure it suits users need. Common user testing.
Early Specification of Physical Architecture – Developing architectures and prototypes early, instead of making it look aesthetically pleasing.

While these changes to the development process are good for creating software for sale, it is not as optimised for creating a larger project with more focus on creating a great experience. A lot of the time this is due to having a larger focus on creating something quickly rather than creating something well. This can often lead to products which would be better with small changes that developers have neglected due to time constraints. 

\section{}
Developers who create or port software for mobile platforms take a large concentration on the user interface. It is often found that this is the most important part of making good software for mobile platforms. Good examples of this come from games that have been ported from one platform to another. One example is Hearthstone which was originally developed for PC, but was ported onto Ipad then Iphone with cross platform functionality \cite{design and development of hearthstone_2017}. Their largest focus was making the game feel smooth for mobile players and still make it familiar to those who had come from playing on PC using a mouse. Another game which was ported from other platforms to mobile devices was Grand Theft Auto vice city. While for the most part the port was a success. It an well on mobile devices one thing that many people complained about was that the interface was difficult to use and did not help the users have an enjoyable experience. \cite{rice_2012}
It should be noted that ports of software will be limited by the hardware of the devices they port to. Storage space and performance in mobile devices is a premium, which will make porting games to a mobile platforms difficult because of potential performance issues. 

\section{}
One difficulty with developing software for mobile platforms is learning new languages. While there are many languages and they all have their own advantages and disadvantages certain platforms of mobile devices only allow the use of certain languages. Apple’s IOS requires developers to make software using objectiveC or Swift. For development companies this is often annoying as it means that programmers will have to learn new languages and potentially new development tools. This process takes time, slows development and increases costs, which is very unappealing for businesses. This could lead to companies with more specific and technical app needs outsourcing app development. 
Android on the other hand, is much more open about the languages it allows the use of. However there are still problems with certain languages. The use of some languages and SDK’s can affect the performance of the software, some may still be unfamiliar to a vast majority of programmers and take time to learn. 
Developers of mobile software should take particular care when choosing which development packages they use as this will have an impact on the end product and how well it sells.




\section{Conclusion}
When developing software for mobile devices, developers should ensure they concentrate on using consistent test driven testing, with a focus upon creating an easy to use intuitive interface. Also ensuring that the development package used it sensible for the goals of the product. Developers who wish to develop software for mobile platforms will often be limited to making a functional sellable product. Developers wishing to port software will often be limited by hardware and storage space. Which leads to questions such as “what about developers who wish to make large scale software for mobile devices?” or “how long will it be till mobile platforms are no longer limited by the hardware which they contain”. 

\bibliographystyle{ieeetran}
\bibliography{references}

\end{document}