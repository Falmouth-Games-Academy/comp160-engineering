% Please do not change the document class
\documentclass{scrartcl}

% Please do not change these packages
\usepackage[hidelinks]{hyperref}
\usepackage[none]{hyphenat}
\usepackage{setspace}
\doublespace

% You may add additional packages here
\usepackage{amsmath}

% Please include a clear, concise, and descriptive title
\title{How does the engineering workflow differ for a video game when trying to meet a particular Internationalisation standard compared that of other types of software?}

% Please do not change the subtitle
\subtitle{COMP160 - Software Engineering Essay}

% Please put your student number in the author field
\author{1606119}

\begin{document}

\maketitle

\abstract{Localisation of software has the capacity to pose many a challenge to developers, as there are so many factors and small details that come into play, with systems such as menu systems and character interaction (in the context of video games) needing to be completely overhauled depending on the local, with dialect and alphabet used there \cite{hogan2004key} sometimes differing massively, these "Internationalisation Standards" have to be followed by the developer if they want their software to be understandable and acceptable in the local they are adapting it for. So my research question is this, "What problems do Internationalisation Standards pose to the localisation of software, and how can the engineering of said software be adapted to overcome them?, to address this question I will utilise my research into this topic area and synthesise this research into a cohesive whole and discuss how engineering processes can be changed or modified, if at all, to improve the development time and quality of the localisation, while still abiding by the standards that are set, and creating a more accessible and better received end product.}

\section{Introduction}

Write your introduction here. A brief introduction is recommended, which should outline key details of the chosen topic and the reviewed papers, motivate the work, and provide a roadmap of key points to the reader. The motivation is quite important here, as essays should have a contribution (i.e., what is the point of the essay, and what does the reader take away from the essay) and the link between the motivation (in the introduction) and the contribution (in the conclusion) should be made clear.

\section{Your section title here}

Write the main body of your essay here. Add more sections if appropriate. You may choose to write about each of your three papers in its own section, or you may choose a different structure. Either way, remember that you are being assessed on technical insight and analysis: it is not enough to merely summarise the contents of the three papers. You must demonstrate the ability to make inferences beyond what is written in the papers, and to draw the three papers together into a single coherent narrative.

Your essay must make a clear recommendation, in terms of which of the three techniques you have reviewed is the best according to whichever metric or metrics you feel is most appropriate. You must justify your choice, backing it up with empirical evidence. However remember that an academic essay is not a murder mystery: you should already have briefly discussed your recommendation in the introduction and in other parts of the essay. Do not save it for a grand reveal at the end.

\section{Conclusion}

Write your conclusion here. The conclusion should do more than summarise the essay, making clear the contribution of the work and highlighting key points, limitations, and outstanding questions. It should not introduce any new content or information.

\bibliographystyle{ieeetran}
\bibliography{references}

\end{document}
