% Please do not change the document class
\documentclass{scrartcl}

% Please do not change these packages
\usepackage[hidelinks]{hyperref}
\usepackage[none]{hyphenat}
\usepackage{setspace}
\doublespace

% You may add additional packages here
\usepackage{amsmath}
\usepackage{graphicx}
\graphicspath{{D:\natha\Documents\University\Units\COMP160 - Software Engineering}}

% Please include a clear, concise, and descriptive title
\title{How have AAA games adapted to the hearing impared and how can they improve?}

% Please do not change the subtitle
\subtitle{COMP160 - Software Engineering Essay}

% Please put your student number in the author field
\author{1600689}

\begin{document}

\maketitle

\abstract{This paper will consider how the gaming industry and how different AAA games accommodate for the hearing-impaired population. During my research, I have found that a few different methods have been used to try and make the experience of gaming more approachable by the hearing impaired, however I feel that these methods are not quite being used to their full potential when used currently in the industry. I will also consider new ways on how these games can be made to be more accessible by the people considered to be hard of hearing and deaf and how it could be implemented into the current industry.}

\section{Introduction}

I will start by looking at what has been tried within the industry to accommodate for the hearing-impaired within both the software that has been produced and the hardware that has been made to adapt to different levels of impairment. I will consider how these methods have been implemented into current AAA games to increase the accessibility of these games and the systems they are on. I will then discuss how new technology and different ways of portraying information can be used to better help the hearing-impaired to engage and fully enjoy the video games that they are playing.

\section{Software}

When looking at support for the hearing impaired there are many different methods within the software that people have used to try and make video games accessible, one of these methods are different variations on closed captioning with variances on how the text should look and when it should be shown on screen \cite{Coutinho} such as on a syllable-by-syllable basis \cite{Sargent} another looking at how the rate of how the speech is transcribed at a far more efficient rate allowing for extremely quick captioning with the possibility of being used in live events to do with the video game \cite{Lasecki}. Another route being looked into is what part of the different sounds within a game need to be better presented to the hearing-impaired person to help them understand what is going on in the game \cite{Hiraga}
Another method that has been used in multiple different cases is the creation of different animations to portray Sign Language notation \cite{Bouzid} \cite{Namatame}, This would allow the game to be shown in a way that many hearing-impaired people are comfortable seeing and reading.

\section{Hardware}

When looking at different types of hardware for gaming consoles different features have been used to aid the hearing-impaired while gaming such as changing light bars on the controller to represent different changes in the game.

\includegraphics[scale=0.80]{PS4ControllerLightBar}
%[URL: https://www.cnet.com/products/sony-playstation-4-slim/preview/]

Another method that has been explored is using a head mounted display similar to an Oculus Rift to display more detail to the hearing-impaired players \cite{Jones}. A similar approach looks at how a smartwatch could also be used to give the player additional information about the game \cite{Mielke}.
One method quite widely used is the feature of `Dualshock' or `Rumble' which is said to be useful as it is a practical way \cite{Distante} of getting certain information about the game such as quick and sudden movement across to the player.

\begin{center}
	\includegraphics[scale=0.57]{Dualshock}
\end{center}
%URL: http://firststopstradingltd.co.uk/index.php?route=product/product&path=439_441_453&product_id=2229
\section{New Ways to Develop Tecnology for the Hearing-Impaired}

There are many new different types of technology being created and developed so that there are better ways to get information across to the hearing-impaired such as using smartphones and smartwatches \cite{Mielke} to be used in conjunction with other systems to be able to give more in-depth information about the videogame that the user might be playing. New information about what the hearing-impaired are most likely to understand \cite{Lawrence} is also being incorporated into new technology being made to make it as effective as possible to make games more accessible to different people.

\section{How These Methods are Used in AAA Games}

Currently not many AAA games use many of the method discussed above to help make the game more accessible to the hearing-impaired such as Sign Language notation or additional hardware such as a smartwatch or a head mounted display, however most AAA games do have closed captioning for most if not all of their spoken dialogue in the game allowing for the hearing-impaired to have a general understanding of what is being said within the game, one example of this is in the game Half Life 2 \cite{Coutinho}. In terms of different hardware most different gaming systems, with the exception of laptops and PCs, have a rumble feature built within its controllers also with additional features such as coloured LEDs within the controller, however I don’t think that many AAA games are making the most out of the potential that it has in making the game more accessible to the hearing-impaired only using either feature more for aesthetic purposes.

\section{Conclusion}

Overall there are many existing methods to make different types of information accessible to the hearing-impaired and there are many new way currently being created as well, however AAA games seem to purely use closed captioning assuming that it will be enough for the hearing-impaired to understand however it has been show that additional information and potentially a different way of portraying how the dialogue is shown can greatly increase how accessible the game is for the hearing-impaired and how immersive they find the experience.

\bibliographystyle{ieeetran}
\bibliography{references}

\end{document}
