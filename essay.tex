% Please do not change the document class
\documentclass{scrartcl}

% Please do not change these packages
\usepackage[hidelinks]{hyperref}
\usepackage[none]{hyphenat}
\usepackage{setspace}
\doublespace

% You may add additional packages here
\usepackage{amsmath}

% Please include a clear, concise, and descriptive title
\title{Video Game Design and implementation for a low mobility user compared to web design}

% Please do not change the subtitle
\subtitle{COMP160 - Software Engineering Essay}

% Please put your student number in the author field
\author{1507729}

\begin{document}

\maketitle

\abstract{Video games touch hundreds of thousands of lives everyday, from mobile gaming to console releases and beyond. It's become one of the fastest growing creative sectors and continues to grow. This doesn't meaning that everyone gets to enjoy games however despite how readily available they are. This paper will be covering the the state of accessibility within the video games industry and how it compares against website development guidelines and requirements.}


\section{Introduction}

This paper will look at the key differences between the video game industry and websites, regarding accessibility. The video games industry is an ever expanding market, which has a huge audience at approximately 31.6 billion people within the UK alone. \cite{UKIE2017Games} Looking at the figures of usage on the internet and the volume at which it is used, along with the requirements which are enforced to make a a website valid. \cite{world2017internet} \cite{caldwell2008web} It's clear to see that the video games industry is growing rapidly but outside of many regulations which could potentially cause an a decline in independent game projects (indie games) as implementing all these regulations could be well beyond the programmers skill level.

\section{Accessiblity for in Video Games}

It would be unfair to say however that all video games lack any kind of accessibility features, a very recent example is Nier Automata \cite{Platinum2017Nier}, which features an easy mode which allows users with lower mobility skills to still be able to play the game through the use of an automatic response system. It will automatically perform actions like dodging and attacking to a limited extent making the game much easier for people who can't react quickly to various prompts. Many other games have similar systems, however this is not true for a vast majority of AAA releases. 

Video game accessibility is considered from a legal point of view as either impractical due to expense and skills required to meet accessibility laws, unfeasible due to the technology required or would mean significant  changes to the core concepts or quality of the game itself. \cite{powers2015video} It's interesting to note that while there are methods that exist to make games more accessible currently that they aren't required by law. It is worth noting that accessibility laws are often reviewed as to make sure that they haven't become redundant. 

Though it's possible that a new baseline for accessibility features could be pushed forward though this doesn't currently seem to be the case an example being a toggle choice for quick time events so that players with low mobility aren't forced to fail repeatedly halting their progress and enjoyment of the game, there are already existing guidelines \cite{game2012accessability} that even to the basic level could be looked at.

\section{How does it compare to websites}

Looking at websites by comparison to video games, it's readily apparent that the requirements are far more strict for websites than those for video games. Websites have many laws stating they must be accessible or the website owners could face lawsuits for discrimination. \cite{caldwell2008web} 

It's also worth noting that many companies have to make note of these guidelines when creating the initial design of the website if they are outsourcing it \cite{lawson2005web}. Though website laws change from country to country the core principles don't have much variation and the guidelines remain the same. 

One such example of the guidelines established by W3C \cite{wcag2017quick}, is as shown in 2.1.1 Keyboard stating ''All functionality of the content is operable through a keyboard interface without requiring specific timings for individual keystrokes, except where the underlying function requires input that depends on the path of the user's movement and not just the endpoints.'' Which the key point to take away is the specification that there shouldn't be any specific timings required to complete the forms.

\section{Conclusion}
Looking at the difference between Web development and Game Design it's clear that the guidelines are itereative and are being updated and researched so they are likely to change in the coming years, however it's worth noting the distance that we have made towards accessible games and the work that could still be done. The games industry may lack the technology and skills currently to produce games that encompass all forms of accessiblity but clearly some games do make considerations and add these features. At an indie game level it's obvious that meeting this requirments simply may be impossible but for triple A companies, investing a reletively small amount of staff and money compared to the total cost of the project to include these features. Though the only way to truly know if a game is accessible is to have a usability study. \cite{juicy2005valid} 

For future consideration the creation of a rating system like the Pan European Game Information (PEGI) rating system \cite{pegi2017ratings} to rate games on their accessiblity might also encourage and promote disability awareness for companies.

\bibliographystyle{ieeetran}
\bibliography{references}


\end{document}
