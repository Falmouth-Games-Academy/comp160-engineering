% Please do not change the document class
\documentclass{scrartcl}

% Please do not change these packages
\usepackage[hidelinks]{hyperref}
\usepackage[none]{hyphenat}
\usepackage{setspace}
\doublespace

% You may add additional packages here
\usepackage{amsmath}

% Please include a clear, concise, and descriptive title
\title{Video Game Design and implementation for a low mobility user compared to web design}

% Please do not change the subtitle
\subtitle{COMP160 - Software Engineering Essay}

% Please put your student number in the author field
\author{1507729}

\begin{document}

\maketitle

\abstract{Please include an abstract of at most 100 words (these do not count towards your word count).}


\section{Introduction}

This paper will look at the key differences between the video game industry and websites, regarding accessibility. The video games industry is an ever expanding market, which has a huge audience at approximately 31.6 billion people within the UK alone. \cite{UKIE2017Games} Looking at the figures of usage on the internet and the volume at which it is used, along with the requirements which are enforced to make a a website valid. \cite{world2017internet} \cite{caldwell2008web} It's clear to see that the video games industry is growing rapidly but outside of many regulations which could potentially cause an a decline in independent game projects as implementing all these regulations could be well beyond the programmers skill level.

\section{Accessiblity for in Video Games}

It would be unfair to say however that all video games lack any kind of accessibility features, a very recent example is Nier Automata \cite{Platinum2017Nier}, which features an easy mode which allows users with lower mobility skills to still be able to play the game through the use of an automatic response system. It will automatically perform actions like dodging and attacking to a limited extent making the game much easier for people who can't react quickly to various prompts. Many other games have similar systems, however this is not true for a vast majority of AAA releases. 

Video game accessibility is considered from a legal point of view as either impractical due to expense and skills required to meet accessibility laws, unfeasible due to the technology required or would mean significant  changes to the core concepts or quality of the game itself. \cite{powers2015video} It's interesting to note that while there are methods that exist to make games more accessible currently that they aren't required by law. It is worth noting that accessibility laws are often reviewed as to make sure that they haven't become redundant. Though it's possible that a new baseline for accessibility features could be pushed forward though this doesn't currently seem to be the case.

\section{How does it compare to websites}
Looking at websites by comparison to video games, it's very quickly clear that the requirements are far more strict than that of video games with laws stating they must be accessible or they could face lawsuits for discrimination. \cite{caldwell2008web} It's also worth noting that web designers have to make note of these guidelines when creating the initial design of the website. Though website laws change from country to country the core principles don't have much variation.

\section{Conclusion}


\bibliographystyle{ieeetran}
\bibliography{references}


\end{document}
