% Please do not change the document class
\documentclass{scrartcl}

% Please do not change these packages
\usepackage[hidelinks]{hyperref}
\usepackage[none]{hyphenat}
\usepackage{setspace}
\doublespace

% You may add additional packages here
\usepackage{amsmath}

% Please include a clear, concise, and descriptive title
\title{How can action games accommodate for hearing impaired gamers regarding conveying information?}

% Please do not change the subtitle
\subtitle{COMP160 - Software Engineering Essay}

% Please put your student number in the author field
\author{1607934}

\begin{document}

\maketitle

\abstract
Video games have been regarded as a major source of entertainment for countless of people. However, it has proven difficult for impaired individuals of varying kinds to enjoy or even play said games. As a result, this paper has analysed and addressed the various methods that can be implemented in action games to accomodate for hearing impaired gamers, both to improve their experience and generally provide more accessibility for a wider range of players. This has been done with aid from previous research and by discussing potential alternatives.

\section{Introduction}
It is easy to forget the importance of adapting games into meeting specific needs. While the majority of current games' visual and auditory qualities have improved over time, the same cannot be said for their accessibilities \cite{McPheron}. Game accessibility involves creating features and strategies that aid impaired individuals' game experience, being able to truly play games even when under limiting conditions \cite{Bierre}. Indeed, widely known games such as \textbf{Half-Life 2} provides all kinds of visual and auditory feedback \cite{Denise} \cite{Coutinho}. As a result, one would expect current games to have expanded on them and increased the amount of accessibility options. Unfortunately, this is not the case for the majority. Therefore, this paper aims to analyse and discuss the importance of accessibility for the hearing impaired, relevant strategies from previous research, such as \textit{interaction feedback}, \textit{communicating threat}, etc., as well as an example of a third-party modification suited for the hearing impaired, which will feed into the conclusion of the research question \cite{Arch}. Additionally, this paper will focus primarily on action games, and explain the varying options they can provide to aid hearing impaired players in having an enjoyable and immersive experience.

\section{How important is accessibility?}
Simply making games `usable' for the sake of accessibility is not enough, as they have to be both enjoyable and able to provide the same level of playability that able-bodied players receive. Indeed, one can easily measure the amount of time a player needs and uses in a level, or observe if the player is able to successfully complete said level \cite{Arch}. However, this does not suffice, as games have to evoke varying emotions, something that hearing impaired players can easily miss by, for instance, failing to follow cutscenes that contain crucial plot information conveyed through speech, ultimately leaving them disappointed and unsatisfied. Strong, empathetic emotions influence people by allowing them to form their own responses and involve in critical thinking of virtual solutions. This forms a strong interactive bond between the player and the game - or even between two players, such as in multiplayer games. Exclusion from such emotionally rich content can lead to social performance issues and in turn possibly affect cognitive development \cite{Sebanz} \cite{Chartrand}.

\section{Auditory Classes and Strategies}
Various classes that exist in games have been discussed. These represent what features of the sound effects were used to convey information, such as, \textbf{speech}, \textbf{redundancies} and their complementary strategies which involve \textit{interaction feedback} (informing that an action was completed), \textit{communicating threat} (informing nearby threats), including many more \cite{Flav} \cite{Denise}. These can be used to evaluate a game's auditory framework and can serve as basic guidelines for game developers to determine how they can make their games as accessible as possible for the hearing impaired. For instance, regarding \textit{interaction feedback} in \textbf{XIII}, the player understands that their actions are or have been executed through visual redundancies, allowing them to perceive and understand what is happening around them. \textit{Communicating threat} is another notable strategy. A relevant example taken from \textbf{Call of Duty: Modern Warfare 2}, where the player is informed of an imminent grenade explosion by the use of \textbf{speech} and \textbf{redundancies}, both by showing them a grenade icon indicating its location and character dialogue yelling at the player to  ``\textit{Get down!}'', although the latter not being as reliable for the hearing impaired - unless subtitles or closed captions exist within the game, which are other examples of visual support for hearing impaired players \cite{Denise}.

\section{Doom3[CC]}
\textbf{Doom} is a first-person shooter that was released by id Software in 1993. It is considered to be one of the most important and influential titles in video game history, laying the foundations of the now ever so popular first-person shooter genre. Naturally, it gained several sequels - one of which will be discussed, \textbf{Doom 3}. \textbf{Doom 3} in particular has a third-party modification which is available to install if chosen to, simply named Doom3[CC]. Evidently, it provides closed captioning, although with an addition of a visual sound radar which shows the approximate direction and distance of a sound source. This is a worthy example of communal contribution as a result of Doom's software adaptability and relative ease of integration. 


\section{Recommendations}


\hspace{6mm}1.	\textbf{Basic guidelines}

\hspace{-4mm}As mentioned above, the various classes and strategies can be used as basic guidelines to guide game developers into shaping their games with accessibility options. There are multiple features that can be implemented to aid hearing impaired players, such as:

\hspace{-4mm}\textbf{Closed captions}

\hspace{-4mm}Captions for speech, sounds and music. This provides an added layer of visual support alongside regular subtitles, as it can be used to indicate sources of various sounds and can be expanded upon, such as by positioning the source through the caption itself, e.g. (North) or position of the screen, as well as captioning action-packed events, e.g. (Explosion) \cite{Yuan}.

2.	\textbf{Increase and further develop accessibility tools}

\hspace{-4mm}Doom3[CC] is a great addition to the otherwise lacking set of accessibility tools. Considering how much such technologies have improved over the course of time, this specific topic requires and deserves increased attention as a result of a growing industry that is games \cite{Fromme}.

\section{Conclusion}

As the gaming industry and audience grows, the bigger the need for accessibility options - both to increase game immersion and avoid the risk of social exclusion for hearing impaired players. While there are a few action games, both old and new, indie and AAA, with accessibility features, they are not nearly enough. Action games can accommodate for hearing impaired gamers by expanding and promoting the use of modifiable programs that aid the hearing impaired, such as Doom3[CC]. Moreover, providing games with subtle features that aid both impaired and able-bodied individuals, such as redundancies, among many others, deserves further discussion and research.

\bibliographystyle{ieeetran}
\bibliography{references}

\end{document}

