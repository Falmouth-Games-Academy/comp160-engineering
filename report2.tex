% Please do not change the document class
\documentclass{scrartcl}

% Please do not change these packages
\usepackage[hidelinks]{hyperref}
\usepackage[none]{hyphenat}
\usepackage{setspace}
\doublespace

% You may add additional packages here
\usepackage{amsmath}

% Please include a clear, concise, and descriptive title
\title{Some of the Many Things I Need to do to Become a Competent Programmer \\ (Hopefully)}

% Please do not change the subtitle
\subtitle{COMP160 - CPD Report}

% Please put your student number in the author field
\author{1600689}

\begin{document}

\maketitle

\section{Introduction}

Overall my career goals are still quite broad as I feel there is a lot more for me to learn before I can choose what I want to specialise into as there are many different areas that would be interesting to consider. One skill that that I have noticed that needs improving is \textit{version control on SVN}, another skill would be the use of \textit{Blueprints in Unreal Engine 4}, a third skill that needs improving is \textit{Pair-Programming and when its needed}, a fourth skill that needs improving is \textit{Applying Functionality to Textures and Materials in Unreal Engine 4}, and the fifth skill that I believe needs improving is \textit{Clearly Communicating as a Team}.

\section{Version Control - SVN}

Over this semester, I have struggled to use version control due to the fact that I am no longer working on my own or in a small group in a project meaning that I have to import and export the work while others are working on the same project, one way I tried to fix this over the project was using the checkout and check in features from within the SVN repository browsers allowing me to synchronize my work to everyone else’s. This however didn’t work exactly how I planned as each time I tried to upload files I ended up replacing others that I didn’t intend to meaning we had to use previous revisions and make sure that all relevant files where in the latest revision. To get better at this I will try to integrate SVN into more group work to make sure I know how to use it fully and properly as it is essential alongside of GitHub for version control. 

\section{Unreal Engine 4 - Blueprints}

During my different projects this study block I have been using blueprints much more frequently as most of my unreal projects are created using blueprints rather than C++ as it functions better on the studio computers. This means I have had to create many different features within blueprints, and while I have succeeded in doing this so far, I feel I need to look at blueprints more closely to better learn how they fit together so I can create more complex code within blueprints more efficiently. This will allow the quality of my code to also increase, diminishing the amount of repeated code within the blueprints I create and generally making my code much neater and more maintainable for if it needs to be changed in any way. This could also make my code much more adaptable for different uses within the same project if needed.

\section{Pair-Programming - When it is Needed}

When programming for my different pieces of work such as my COMP140 game when working on torque and rotations of a static object as I found myself occasionally getting frustrated at my code when it wasn’t functioning as it should and I ended up spending a lot of time going through my code missing obvious mistakes that I had made, however these cases could have been solved much faster if I had pair-programmed with somebody else as they would have been able to see my mistakes more quickly than I could when going through it. This would allow me to work more efficiently and would make me less stressed when it comes to the programming within my work. Another benefit of pair-programming would be the fact that there can be suggestions for better programming practices such as using functions to call code rather than repeating it, this can also make it easier to create maintainable code for future editing.

\section{Unreal Engine 4 - Applying Functionality to Textures and Materials}

Another skill I need to improve is my ability to apply and manipulate textures and materials while adding different functionality to them within the Unreal Engine 4. I have found that due to my lack of knowledge in this area the progress of my work has slowed significantly as both in my individual work and my team projects it is necessary to understand how these different things work to make a singular coherent project, I have found this is because it gives a clearer image of the engines limitations and capabilities and it allows you to make more within the engine such as sky boxes rather than limiting what I can do to purely coding. Learning how both textures and materials work can also help when adding to the aesthetic of the game as it can give a clearer image of what different items will look like within the game and how they are going to function.

\section{Teamwork - Clear Communication}

When working on my team project I have found that effectively communicating with my team has not been one of my strong points as I sometimes left a meeting not sure of what the team wanted from me over the week due to vague ideas being passed around with no solid ideas being given. This made it hard to know what exactly I needed to be doing on a day to day basis meaning that a lot of my code that I had written had gone to waste as it was no longer necessary within the game as different ideas had come along that where better which the rest of the team had decided to work on. This is incredibly bad as communication is one of the key aspects of working effectively within a team and to make a quality product within shorter periods of time. This is essential to fix as this caused many issues over the past study block with wasted time and incomplete coding, making the team disjointed and the game had to be scaled back to a more basic idea.

\section{Conclusion}

In conclusion over the summer I will look at using SVN with bigger teams on self-directed projects to make sure that everyone can contribute to the project while not overriding existing work, I will also study blueprints closer to make sure I can efficiently create programs within the Unreal Engine 4, When creating new work I will also actively try to pair-program where possible to make my code more efficient, I will also look into more complex functionalities within textures and materials to make sure I have a relevant competency when using them in the Unreal Engine 4 and I will work on my communication skills when working on these self-directed projects to make sure I always know what I need to do when working on a team with others.

\end{document}