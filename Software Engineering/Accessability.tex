% Please do not change the document class
\documentclass[11pt]{scrartcl}

% Please do not change these packages
\usepackage[hidelinks]{hyperref}
\usepackage[none]{hyphenat}
\usepackage{setspace}
\doublespace

% You may add additional packages here
\usepackage{amsmath}

% Please include a clear, concise, and descriptive title+
\title{What is stopping game developers from broadening access to their games? }

\date{January 6, 2017}

% Please do not change the subtitle
\subtitle{COMP160 - Software Engineering}

% Please put your student number in the author field
\author{1605913}



\begin{document}

\maketitle

\abstract{ Proposal}

This paper will discuss whether the games industry should worry about whether their games can be played by people with disabilities and how this decision may impact the industry. It is a big decision to make a game unplayable by an entire facet of people. A company’s decision to do this is based on two considerations: will investing in game accessibility return a profit and will it benefit my company long-term? \cite{bierre2005game}. These decisions are mainly made by the companies' marketing departments in order to target the most profitable groups of people \cite{kalapanidas2009playmancer}. However, this potentially incurs a huge loss. For example, in a US census a quarter of the population was made up by disabled citizens\cite{bierre2005game} (Table 2). So, although this lack of accessibility theoretically has an economic motive, a company incurs three main problems: limiting potential feedback, reduced ability to determine in-game responses on a large scale and a reduced ability to provide input using conventional input devices. However, these inevitably all require considerable investment\cite{yuan2011game}.

\section{Introduction}

This paper will discuss the importance of accessibility in games in the industry. In looking at the three primary reasons as to why game developers do not commonly consider accessibility in their games, such as a lack of game expansion for the blind and deaf, the proposal will be able to be properly evaluated. Furthermore, by considering the three most predominant engineering challenges which deter game developers from broadening access to their games for specific disabilities, this essay can investigate how these problems can potentially be overcome.

\section{Developing Games for the Visually Impaired}

In the games industry haptics, have been a very useful resource to developers. Using haptics, a game can be given a more realistic and immersive feeling\cite{orozco2012role}, but that is not where the only use for haptics lies. 
Haptics becomes incredibly useful for allowing people with visual impairments to play games. Haptics allows a person with a visual impairment to feel the game instead of using visual cues to operate through the game. 
An example of this would be blind hero where, instead of pressing coloured buttons on a controller to match coloured buttons on a screen, the use of a glove with pager motors inside it help to produce stimuli for the participants to play the game.\cite{yuan2008blind}
If a developer decided that they wanted to broaden the accessibility of their game to the visually impaired using haptics, they would need to first check whether their game was suitable for it. This provides a large barrier for a game developer to try and overcome. This is because if a game has a lot of objects that the player needs to follow, using vibrations overload the player with haptic signals and this will make it very difficult for them to play\cite{orozco2012role}. 
It is also a hurdle for developers because finding substitutes for visual representations at the same level as something like graphics for the visually impaired poses a large challenge \cite{yuan2009towards}.
Although making games with content that substitutes visuals is considered very hard it is not impossible. There are a few ways to make a game accessible to those with low-level vision these include:
A high contrast mode, this turns part of the 3D graphics into high contrasted black and white colours on important items for gamers with low vision.
Backpack mode, where items are accessed through a voice activated menu with hierarchies.
Game objects, all game items have a voice feedback and 3D icon sounds and Graphics \cite{westin2004game}
Those are some of the ways that were used to make a game accessible to gamers with low vision, the only drawback to this project is that the researchers had to make their own game because they could not find a 3D graphics game that had a sound interface for the blind. From Table 1 in \cite{westin2004game} it is clear to see that the users found it very easy to use and that the usability was well done.


\section{Developing Games for Gamers with Auditory Disabilities}

In the games industry, quite a few games have been optimised for gamers with deafness such as Half-life 2. This game used closed captioning to allow deaf people the ability to play the game more efficiently, but this feature was not in the game originally. In Half-Life 2 the deaf community of gamers requested that for the second instalment of the game that a closed captioning feature be put into the game. So, for their second instalment Valve decided that this issue would be part of the main development effort. The playtesters for the game were even deaf gamers to improve their results. The main concern that that developer had when adding in accessibility to their games is the impact that it will have on development in terms of time and money. Valve found that this impact was very small, the reason for this is that the script had to write out anyway\cite{bierre2005game}.


\section{Developing Games for Gamers with Mobility Disabilities}

For gamers with Motor disabilities, it can be very hard for them to use traditional controllers such as for an XBOX 360 or PS4. The reason that these controllers are not usable is because the buttons are not spaced out enough and the buttons are too small. For this reason, they cannot keep up with the pace of a game. One way to combat this problem is to make single or multiple switch interfaces, these come in a large size and don’t take a lot of force to activate. The problem for developers is that most of these controllers must be made and tailored to the individual, so cannot be mass produced and are restricted to a limited amount of disabilities \cite{6031883}. Another use of buttons is for people with severe motor disabilities, that must rely on a single button to allow them to play games. This research was taken up by a group of researchers who use GNomon, which is a framework that allows developers to create games that only need one switch to operate. This makes it easier for developers to make games that are accessible to people with motor disabilities as they can as there is already a framework for them to work on this will reduce time and money spent.\cite{7325502}
Another interesting way that game developers can allow games to be played by people with severe mobility issues is with eye tracking technology, this must be done through an external device such as the Kinect for XBOX. This is called OAK (Observation and Access with Kinect) \cite{7803137}.


\section{Conclusion}

In conclusion, while it may be taxing on a developers budget and time to develop a more accessible approach to their game. The advantages of adopting this approach are numerous and should be practised as some of these approaches are very easy to implement and adapt to different systems. While not all games can be adapted for use for those with disabilities, the use of closed captions and haptics are very easy concepts to include in games.

\bibliographystyle{ieeetran}
\bibliography{References}

\end{document}