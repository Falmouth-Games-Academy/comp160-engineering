% Please do not change the document class
\documentclass[11pt]{scrartcl}

% Please do not change these packages
\usepackage[hidelinks]{hyperref}
\usepackage[none]{hyphenat}
\usepackage{setspace}
\doublespace

% You may add additional packages here
\usepackage{amsmath}

% Please include a clear, concise, and descriptive title
\title{Should Whether Games Are Accessible Matter To Companies In The Games Industry?}

\date{January 6, 2017}

% Please do not change the subtitle
\subtitle{COMP160 - Software Engineering}

% Please put your student number in the author field
\author{1605913}



\begin{document}

\maketitle

\abstract{ Proposal}

This Paper will discuss whether the games industry should worry whether their games can be played by people with disabilities and how this decision will impact the industry. It is a big decision whether to make a game not playable by a certain group of people. A company’s decision to do this is based off two questions, will this investment in making my game accessible return with a profit and how will this investment benefit my company in the long run? \cite{bierre2005game}. These decisions ae made mainly by the marketing department of these companies to target certain groups of people which will make them the most money\cite{kalapanidas2009playmancer}. This is a huge loss because in a US census the percentage of the population which were disabled made up almost a quarter of the population\cite{bierre2005game} (Table 2). Although the reason for this lack of accessibility in games is not without reason, a company has three problems pertaining to this subject. Not being able to receive feedback, not being able to determine in-game responses and not being able to provide input using conventional input devices. These all require considerable investment\cite{yuan2011game}.


\bibliographystyle{ieeetran}
\bibliography{References}

\end{document}